%
% \usepackage{tikz}
% \usepackage{xifthen}
% \usepackage{fp}
\usetikzlibrary{calc}
%
%       new commands (-> Diagram-package)
%           D.Kats, September 2011
%
% \bdiag[s] start diagram. If s is given, the H diagram is in the middle (otherwise it is shifted to the vacuum)
%
% \dT[<order>]{<exc.level>}{<node>} -> T_<exc.level>^{(<order>)}
% node-names are generated as <node>1, <node>2, <node>3, ...
% can be used without <order> (\dT{<exc.level>}{<node>})
% same with \dU (will print U_<exc.level> as label)
% for excitation operators without label use \dTs
% One can customize labels (see e.g. how \dU is defined)
% \dTd, \dTds, \dUd are T^{\dagger}, {}^{\dagger}, U^{\dagger}
%
% \dTdv[<order>]{<exc.level>}{<node>} : draw vacuum explicitly (\tau^\dagger) 
% \dTv[<order>]{<exc.level>}{<node>} -> \tau
%
% \dF{<node>} -> F
% \dFs{<node>} : one-electron operator without label
% \dX{<node>} : one-electron perturbation (X as label)
% \dHone[<name>]{<node>} : one-electron operator with label <name> 
% \dW{<left node>}{<right node>} -> W
% \dWs{<left node>}{<right node>} : two-electron operator without label
% \dXtwo{<left node>}{<right node>} : two-electron perturbation (X as label)
% \dHtwo[<name>]{<left node>}{<right node>} : two-electron operator with label <name>
%
% \dline[<index>]{<from node>}{<to node>} ->    "--->" (if <index> given - write <index> to the right of the line)
% \dcurve[<index>]{<from node>}{<to node>} ->   curved "--->" 
% \dcurver[<index>]{<from node>}{<to node>} ->   curved "--->" (reverse bend)
% \dcurcur{<node1>}{<node2>} ->     cycled curved "--->"
%  left vacuum-node for <node1> is called <node1>v1
%  right -------------"----------------   <node1>v2
%
% \dmovex{<value>} -> move W or F horizontally
% \dmoveT{<value>} -> move T horizontally
% \dmovac{<value>} -> move vacuum (and daggers) vertically 
% \dmoveTd{<value>} -> move T^\dagger horizontally
%
% \dscale{<value>} -> scale size of diagrams with <value>
%
% \dname{<text>} -> write <text> over the diagram
% \dtext{<text>} -> write <text> in the diagram
%
% change exoper_line_save, exvac_line_save, hoper_line_save, ph_line_save
%  in order to change excitation operator, explicit vacuum, H-operator, or p/h line styles.
%
% Old commands (useful for custom node-names):
% \dTone{<node>} -> T_1
% \dTtwo{<left node>}{<right node>} -> T_2
% \dTthr{<left node>}{<middle node>}{<right node>} -> T_3
%
% \dTone[<order>]{<node>} -> T_1^{(<order>)}
% \dTtwo[<order>]{<left node>}{<right node>} -> T_2^{(<order>)}
% \dTthr[<order>]{<left node>}{<middle node>}{<right node>} -> T_3^{(<order>)}
%
% same with \dUone, \dUtwo, \dUthr (will print U_1, U_2 or U_3)
% for excitation operators without label use \dTones, \dTtwos, \dTthrs
% One can customize labels (see e.g. how \dUone is defined)


\edef\xcoord{11} \edef\xcoor{11} \edef\xcoorbeg{11} \edef\xcoorend{11} \edef\result{11} \edef\ycoor{11} \edef\ycoord{11} \edef\sc{11} 
\FPset\sc{1} %default scale-value
\FPeval\xcoord{clip(-1*\sc)} \FPeval\ycoord{clip(-1*\sc)} 
\FPset\xcoor{\xcoord} \FPeval\ycoor{clip(\ycoord+1*\sc)} 
\FPset\xcoordg{\xcoord}
\FPset\xcoorH{\xcoord}
\FPeval\yvac{clip(\ycoor+0.5*\sc)} \FPeval\xvac{clip(0.25*\sc)}
\newcounter{stoer} 
\tikzset{exoper_line_save/.style={very thick, solid}}
\tikzset{exvac_line_save/.style={very thick, dotted}}
\tikzset{hoper_line_save/.style={very thick, dashed}}
\tikzset{ph_line_save/.style={->,>=stealth,thin}}
\tikzset{exoper_line/.style={exoper_line_save}}
\tikzset{hoper_line/.style={hoper_line_save}}
\tikzset{ph_line/.style={ph_line_save}}

\newcommand{\bdiag}[1][]{
  \begin{tikzpicture}
    \ifthenelse{\isempty{#1}}
      {}
      {%symmetric diagram
        \FPeval\yvac{clip(\ycoor+1*\sc)}
        \FPeval\xvac{clip(0.5*\sc)}
      }
}
\newcommand{\ediag}{\end{tikzpicture}}

\newcommand{\dAmpOne}[3][]{\FPeval\xcoorbeg{clip(\xcoor+0.5*\sc)} 
\FPeval\xcoorend{clip(\xcoorbeg+0.5*\sc)} 
\draw[very thick](\xcoorbeg,\ycoord) -- (\xcoorend,\ycoord);
\FPeval\result{clip(\xcoorend+0.35*\sc)} 
\ifthenelse{\isempty{#1}}  
{\FPeval\xcoor{\xcoorend}}
{\node at (\result,\ycoord){#1};
\FPeval\xcoor{clip(\xcoorend+0.25*\sc)}} 
\FPeval\result{clip((\xcoorbeg+\xcoorend)/2)} 
\node[inner sep=0pt,minimum size=0pt] (#3) at  (\result,\ycoord) {};
\FPeval\xx{clip(\xcoorbeg-\xvac)} 
\node[inner sep=0pt,minimum size=0pt] (#3v1) at (\xx,\yvac) {}; 
\FPeval\xx{clip(\xcoorbeg+\xvac)} 
\node[inner sep=0pt,minimum size=0pt] (#3v2) at (\xx,\yvac) {};
\ifthenelse{\isempty{#2}}  {}  
{ 
\FPeval\result{clip((\xcoorend+\xcoorbeg)/2)} 
\setcounter{stoer}{#2} 
\node[inner sep=0pt,minimum size=0pt] at (\result,\ycoord){{\footnotesize\slshape\sffamily \Roman{stoer}}}; }
}

\newcommand{\dAmpTwo}[4][]{\FPeval\xcoorbeg{clip(\xcoor+0.5*\sc)} 
\FPeval\xcoorend{clip(\xcoorbeg+1*\sc)} 
\draw[very thick](\xcoorbeg,\ycoord) -- (\xcoorend,\ycoord);
\FPeval\result{clip(\xcoorend+0.35*\sc)} 
\ifthenelse{\isempty{#1}}  
{\FPeval\xcoor{\xcoorend}}
{\node at (\result,\ycoord){#1};
\FPeval\xcoor{clip(\xcoorend+0.25*\sc)}}
\node[inner sep=0pt,minimum size=0pt] (#3) at (\xcoorbeg,\ycoord) {};
\node[inner sep=0pt,minimum size=0pt] (#4) at (\xcoorend,\ycoord) {};  
\FPeval\xx{clip(\xcoorbeg-\xvac)} 
\node[inner sep=0pt,minimum size=0pt] (#3v1) at (\xx,\yvac) {}; 
\FPeval\xx{clip(\xcoorbeg+\xvac)} 
\node[inner sep=0pt,minimum size=0pt] (#3v2) at (\xx,\yvac) {};
\FPeval\xx{clip(\xcoorend-\xvac)} 
\node[inner sep=0pt,minimum size=0pt] (#4v1) at (\xx,\yvac) {};
\FPeval\xx{clip(\xcoorend+\xvac)} 
\node[inner sep=0pt,minimum size=0pt] (#4v2) at (\xx,\yvac) {};
\ifthenelse{\isempty{#2}}  {}  
{ 
\FPeval\result{clip((\xcoorend+\xcoorbeg)/2)} 
\setcounter{stoer}{#2} 
\node[inner sep=0pt,minimum size=0pt] at (\result,\ycoord){{\footnotesize\slshape\sffamily \Roman{stoer}}}; }
}

\newcommand{\dAmpThr}[5][]{\FPeval\xcoorbeg{clip(\xcoor+0.5*\sc)} 
\FPeval\xcoorend{clip(\xcoorbeg+1*\sc)} 
\draw[very thick](\xcoorbeg,\ycoord) -- (\xcoorend,\ycoord);
\FPeval\result{clip(\xcoorend+0.35*\sc)} 
\ifthenelse{\isempty{#1}}  
{\FPeval\xcoor{\xcoorend}}
{\node at (\result,\ycoord){#1};
\FPeval\xcoor{clip(\xcoorend+0.25*\sc)}}
\node[inner sep=0pt,minimum size=0pt] (#3) at (\xcoorbeg,\ycoord) {};
\FPeval\result{clip((\xcoorbeg+\xcoorend)/2)} 
\node[inner sep=0pt,minimum size=0pt] (#4) at  (\result,\ycoord) {};
\node[inner sep=0pt,minimum size=0pt] (#5) at (\xcoorend,\ycoord) {};  
\FPeval\xx{clip(\xcoorbeg-\xvac)} 
\node[inner sep=0pt,minimum size=0pt] (#3v1) at (\xx,\yvac) {}; 
\FPeval\xx{clip(\xcoorbeg+\xvac)} 
\node[inner sep=0pt,minimum size=0pt] (#3v2) at (\xx,\yvac) {};
\FPeval\xx{clip(\result-\xvac)} 
\node[inner sep=0pt,minimum size=0pt] (#4v1) at (\xx,\yvac) {};
\FPeval\xx{clip(\result+\xvac)} 
\node[inner sep=0pt,minimum size=0pt] (#4v2) at (\xx,\yvac) {};
\FPeval\xx{clip(\xcoorend-\xvac)} 
\node[inner sep=0pt,minimum size=0pt] (#5v1) at (\xx,\yvac) {};
\FPeval\xx{clip(\xcoorend+\xvac)} 
\node[inner sep=0pt,minimum size=0pt] (#5v2) at (\xx,\yvac) {};
\ifthenelse{\isempty{#2}}  {}  
{ 
\FPeval\result{clip((\xcoorend+\xcoorbeg)/2)} 
\setcounter{stoer}{#2} 
\node[inner sep=0pt,minimum size=0pt] at (\result,\ycoord){{\footnotesize\slshape\sffamily \Roman{stoer}}}; }
}



\newcommand{\dAmpdOne}[3][]{\FPeval\xcoorbeg{clip(\xcoordg+0.5*\sc)} 
\FPeval\xcoorend{clip(\xcoorbeg+0.5*\sc)} 
\draw[very thick](\xcoorbeg,\yvac) -- (\xcoorend,\yvac);
\FPeval\result{clip(\xcoorend+0.35*\sc)} 
\ifthenelse{\isempty{#1}}  
{\FPeval\xcoordg{\xcoorend}}
{\node at (\result,\yvac){#1};
\FPeval\xcoordg{clip(\xcoorend+0.25*\sc)}}
\FPeval\result{clip((\xcoorbeg+\xcoorend)/2)} 
\node[inner sep=0pt,minimum size=0pt] (#3) at  (\result,\yvac) {};
\FPeval\xx{clip(\result-\xvac)} 
\node[inner sep=0pt,minimum size=0pt] (#3v1) at (\xx,\ycoord) {};
\FPeval\xx{clip(\result+\xvac)} 
\node[inner sep=0pt,minimum size=0pt] (#3v2) at (\xx,\ycoord) {};
\ifthenelse{\isempty{#2}}  {}  
{ 
\FPeval\result{clip((\xcoorend+\xcoorbeg)/2)} 
\setcounter{stoer}{#2} 
\node[inner sep=0pt,minimum size=0pt] at (\result,\yvac){{\footnotesize\slshape\sffamily \Roman{stoer}}}; }
}

\newcommand{\dAmpdTwo}[4][]{\FPeval\xcoorbeg{clip(\xcoordg+0.5*\sc)} 
\FPeval\xcoorend{clip(\xcoorbeg+1*\sc)} 
\draw[very thick](\xcoorbeg,\yvac) -- (\xcoorend,\yvac);
\FPeval\result{clip(\xcoorend+0.35*\sc)} 
\ifthenelse{\isempty{#1}}  
{\FPeval\xcoordg{\xcoorend}}
{\node at (\result,\yvac){#1};
\FPeval\xcoordg{clip(\xcoorend+0.25*\sc)}}
\node[inner sep=0pt,minimum size=0pt] (#3) at (\xcoorbeg,\yvac) {};
\node[inner sep=0pt,minimum size=0pt] (#4) at (\xcoorend,\yvac) {};  
\FPeval\xx{clip(\xcoorbeg-\xvac)} 
\node[inner sep=0pt,minimum size=0pt] (#3v1) at (\xx,\ycoord) {}; 
\FPeval\xx{clip(\xcoorbeg+\xvac)} 
\node[inner sep=0pt,minimum size=0pt] (#3v2) at (\xx,\ycoord) {};
\FPeval\xx{clip(\xcoorend-\xvac)} 
\node[inner sep=0pt,minimum size=0pt] (#4v1) at (\xx,\ycoord) {};
\FPeval\xx{clip(\xcoorend+\xvac)} 
\node[inner sep=0pt,minimum size=0pt] (#4v2) at (\xx,\ycoord) {};
\ifthenelse{\isempty{#2}}  {}  
{ 
\FPeval\result{clip((\xcoorend+\xcoorbeg)/2)} 
\setcounter{stoer}{#2} 
\node[inner sep=0pt,minimum size=0pt] at (\result,\yvac){{\footnotesize\slshape\sffamily \Roman{stoer}}}; }
}

\newcommand{\dAmpdThr}[5][]{\FPeval\xcoorbeg{clip(\xcoordg+0.5*\sc)} 
\FPeval\xcoorend{clip(\xcoorbeg+1*\sc)} 
\draw[very thick](\xcoorbeg,\yvac) -- (\xcoorend,\yvac);
\FPeval\result{clip(\xcoorend+0.35*\sc)} 
\ifthenelse{\isempty{#1}}  
{\FPeval\xcoordg{\xcoorend}}
{\node at (\result,\yvac){#1};
\FPeval\xcoordg{clip(\xcoorend+0.25*\sc)}}
\node[inner sep=0pt,minimum size=0pt] (#3) at (\xcoorbeg,\yvac) {};
\FPeval\result{clip((\xcoorbeg+\xcoorend)/2)} 
\node[inner sep=0pt,minimum size=0pt] (#4) at  (\result,\yvac) {};
\node[inner sep=0pt,minimum size=0pt] (#5) at (\xcoorend,\yvac) {};  
\FPeval\xx{clip(\xcoorbeg-\xvac)} 
\node[inner sep=0pt,minimum size=0pt] (#3v1) at (\xx,\ycoord) {}; 
\FPeval\xx{clip(\xcoorbeg+\xvac)} 
\node[inner sep=0pt,minimum size=0pt] (#3v2) at (\xx,\ycoord) {};
\FPeval\xx{clip(\result-\xvac)} 
\node[inner sep=0pt,minimum size=0pt] (#4v1) at (\xx,\ycoord) {};
\FPeval\xx{clip(\result+\xvac)} 
\node[inner sep=0pt,minimum size=0pt] (#4v2) at (\xx,\ycoord) {};
\FPeval\xx{clip(\xcoorend-\xvac)} 
\node[inner sep=0pt,minimum size=0pt] (#5v1) at (\xx,\ycoord) {};
\FPeval\xx{clip(\xcoorend+\xvac)} 
\node[inner sep=0pt,minimum size=0pt] (#5v2) at (\xx,\ycoord) {};
\ifthenelse{\isempty{#2}}  {}  
{ 
\FPeval\result{clip((\xcoorend+\xcoorbeg)/2)} 
\setcounter{stoer}{#2} 
\node[inner sep=0pt,minimum size=0pt] at (\result,\yvac){{\footnotesize\slshape\sffamily \Roman{stoer}}}; }
}

\newcommand{\dAmp}[4][]{%for all
\FPeval\xcoorbeg{clip(\xcoor+0.5*\sc)} 
\FPeval\xcoorend{clip(\xcoorbeg+#3*\sc/2)} 
\draw[exoper_line](\xcoorbeg,\ycoord) -- (\xcoorend,\ycoord);
\FPeval\result{clip(\xcoorend+0.35*\sc)} 
% write label if given
\ifthenelse{\isempty{#1}}  
{\FPeval\xcoor{\xcoorend}}
{\node at (\result,\ycoord){#1};
\FPeval\xcoor{clip(\xcoorend+0.25*\sc)}}
% calculate the vertex-distance (singles are a special case)
\ifthenelse{ #3 = 1}
{ \FPeval\xxx{clip((\xcoorend-\xcoorbeg)/2)} 
  \FPset\result{\xcoorbeg}}
{ \FPeval\xxx{clip((\xcoorend-\xcoorbeg)/( #3 - 1 ))} 
  \FPeval\result{clip(\xcoorbeg-\xxx)}}
% set all nodes (and give them names)
\foreach \x in {1,...,#3}
{
  \coordinate (vertex) at ($(\result,\ycoord)+(\x*\xxx,0)$);
  \node[inner sep=0pt,minimum size=0pt] (#4\x) at  (vertex) {};
  \coordinate (vtxvac) at ($(\result,\yvac)+(\x*\xxx,0)$);
  \node[inner sep=0pt,minimum size=0pt] (#4\x v1) at ($(vtxvac)-(\xvac,0)$) {};
  \node[inner sep=0pt,minimum size=0pt] (#4\x v2) at ($(vtxvac)+(\xvac,0)$) {};
}
\node[inner sep=0pt,minimum size=0pt] (#4) at  (#41) {};
\node[inner sep=0pt,minimum size=0pt] (#4v1) at (#41v1) {};
\node[inner sep=0pt,minimum size=0pt] (#4v2) at (#41v2) {};
% set perturbation (if given)
\ifthenelse{\isempty{#2}}  {}  
{ 
\FPeval\result{clip((\xcoorend+\xcoorbeg)/2)} 
\setcounter{stoer}{#2} 
\node[inner sep=0pt,minimum size=0pt] at (\result,\ycoord){{\footnotesize\slshape\sffamily \Roman{stoer}}}; }
}

\newcommand{\dAmpD}[4][]{%for all
\FPeval\xcoorbeg{clip(\xcoordg+0.5*\sc)} 
\FPeval\xcoorend{clip(\xcoorbeg+#3*\sc/2)} 
\draw[exoper_line](\xcoorbeg,\yvac) -- (\xcoorend,\yvac);
\FPeval\result{clip(\xcoorend+0.35*\sc)} 
% write label if given
\ifthenelse{\isempty{#1}}  
{\FPeval\xcoordg{\xcoorend}}
{\node at (\result,\yvac){#1};
\FPeval\xcoordg{clip(\xcoorend+0.25*\sc)}}
% calculate the vertex-distance (singles are a special case)
\ifthenelse{ #3 = 1}
{ \FPeval\xxx{clip((\xcoorend-\xcoorbeg)/2)} 
  \FPset\result{\xcoorbeg}}
{ \FPeval\xxx{clip((\xcoorend-\xcoorbeg)/( #3 - 1 ))} 
  \FPeval\result{clip(\xcoorbeg-\xxx)}}
% set all nodes (and give them names)
\foreach \x in {1,...,#3}
{
  \coordinate (vertex) at ($(\result,\yvac)+(\x*\xxx,0)$);
  \node[inner sep=0pt,minimum size=0pt] (#4\x) at  (vertex) {};
  \coordinate (vtxvac) at ($(\result,\ycoord)+(\x*\xxx,0)$);
  \node[inner sep=0pt,minimum size=0pt] (#4\x v1) at ($(vtxvac)-(\xvac,0)$) {};
  \node[inner sep=0pt,minimum size=0pt] (#4\x v2) at ($(vtxvac)+(\xvac,0)$) {};
}
\node[inner sep=0pt,minimum size=0pt] (#4) at  (#41) {};
\node[inner sep=0pt,minimum size=0pt] (#4v1) at (#41v1) {};
\node[inner sep=0pt,minimum size=0pt] (#4v2) at (#41v2) {};
% set perturbation (if given)
\ifthenelse{\isempty{#2}}  {}  
{ 
\FPeval\result{clip((\xcoorend+\xcoorbeg)/2)} 
\setcounter{stoer}{#2} 
\node[inner sep=0pt,minimum size=0pt] at (\result,\yvac){{\footnotesize\slshape\sffamily \Roman{stoer}}}; }
}

\newcommand{\dHone}[2][]{\FPeval\xcoorbeg{clip(\xcoorH+0.5*\sc)}
\FPeval\xcoorend{clip(\xcoorbeg+0.8*\sc)} 
\draw[hoper_line](\xcoorbeg,\ycoor) -- (\xcoorend,\ycoor);
\node at (\xcoorend,\ycoor){$\bf \times$};
\FPeval\result{clip(\xcoorend+0.35*\sc)} 
%name of operator
\ifthenelse{\isempty{#1}}
{\FPeval\xcoorH{\xcoorend}}
{\node at (\result,\ycoor){#1};
\FPeval\xcoorH{clip(\xcoorend+0.25*\sc)}}
\node[inner sep=0pt,minimum size=0pt] (#2) at (\xcoorbeg,\ycoor){}; 
\FPeval\xx{clip(\xcoorbeg-\xvac/2)} 
\node[inner sep=0pt,minimum size=0pt] (#2v1) at (\xx,\yvac) {}; 
\FPeval\xx{clip(\xcoorbeg+\xvac/2)} 
\node[inner sep=0pt,minimum size=0pt] (#2v2) at (\xx,\yvac) {};
}

\newcommand{\dHtwo}[3][]{ \FPeval\xcoorbeg{clip(\xcoorH+0.5*\sc)}
\FPeval\xcoorend{clip(\xcoorbeg+1*\sc)} 
\draw[hoper_line](\xcoorbeg,\ycoor) -- (\xcoorend,\ycoor); 
\FPeval\result{clip(\xcoorend+0.35*\sc)} 
%name of operator
\ifthenelse{\isempty{#1}}
{\FPeval\xcoorH{\xcoorend}}
{\node at (\result,\ycoor){#1};
\FPeval\xcoorH{clip(\xcoorend+0.25*\sc)}}
\node[inner sep=0pt,minimum size=0pt] (#2) at (\xcoorbeg,\ycoor) {};
\node[inner sep=0pt,minimum size=0pt] (#3) at (\xcoorend,\ycoor) {};
\FPeval\xx{clip(\xcoorbeg-\xvac/2)} 
\node[inner sep=0pt,minimum size=0pt] (#2v1) at (\xx,\yvac) {};
\FPeval\xx{clip(\xcoorbeg+\xvac/2)} 
\node[inner sep=0pt,minimum size=0pt] (#2v2) at (\xx,\yvac) {}; 
\FPeval\xx{clip(\xcoorend-\xvac/2)} 
\node[inner sep=0pt,minimum size=0pt] (#3v1) at (\xx,\yvac) {};
\FPeval\xx{clip(\xcoorend+\xvac/2)} 
\node[inner sep=0pt,minimum size=0pt] (#3v2) at (\xx,\yvac) {};
} 

\newcommand{\dTone}[2][]{\dAmpOne[$_{T_1}$]{#1}{#2}}
\newcommand{\dTones}[2][]{\dAmpOne{#1}{#2}}
\newcommand{\dUone}[2][]{\dAmpOne[$_{U_1}$]{#1}{#2}}

\newcommand{\dTtwo}[3][]{\dAmpTwo[$_{T_2}$]{#1}{#2}{#3}}
\newcommand{\dTtwos}[3][]{\dAmpTwo{#1}{#2}{#3}}
\newcommand{\dUtwo}[3][]{\dAmpTwo[$_{U_2}$]{#1}{#2}{#3}}

\newcommand{\dTthr}[4][]{\dAmpThr[$_{T_3}$]{#1}{#2}{#3}{#4}}
\newcommand{\dTthrs}[4][]{\dAmpThr{#1}{#2}{#3}{#4}}
\newcommand{\dUthr}[4][]{\dAmpThr[$_{U_3}$]{#1}{#2}{#3}{#4}}

\newcommand{\dTdone}[2][]{\dAmpdOne[$_{T^{\dagger}_1}$]{#1}{#2}}
\newcommand{\dTdones}[2][]{\dAmpdOne{#1}{#2}}
\newcommand{\dUdone}[2][]{\dAmpdOne[$_{U^{\dagger}_1}$]{#1}{#2}}

\newcommand{\dTdtwo}[3][]{\dAmpdTwo[$_{T^{\dagger}_2}$]{#1}{#2}{#3}}
\newcommand{\dTdtwos}[3][]{\dAmpdTwo{#1}{#2}{#3}}
\newcommand{\dUdtwo}[3][]{\dAmpdTwo[$_{U^{\dagger}_2}$]{#1}{#2}{#3}}

\newcommand{\dTdthr}[4][]{\dAmpdThr[$_{T^{\dagger}_3}$]{#1}{#2}{#3}{#4}}
\newcommand{\dTdthrs}[4][]{\dAmpdThr{#1}{#2}{#3}{#4}}
\newcommand{\dUdthr}[4][]{\dAmpdThr[$_{U^{\dagger}_3}$]{#1}{#2}{#3}{#4}}

%general excitations
\newcommand{\dT}[3][]{\dAmp[$_{T_#2}$]{#1}{#2}{#3}}
\newcommand{\dTs}[3][]{\dAmp{#1}{#2}{#3}}
\newcommand{\dU}[3][]{\dAmp[$_{U_#2}$]{#1}{#2}{#3}}
\newcommand{\dTv}[3][]{
\tikzset{exoper_line/.style={exvac_line_save}}
\dAmp{#1}{#2}{#3}
\tikzset{exoper_line/.style={exoper_line_save}} }

\newcommand{\dTd}[3][]{\dAmpD[$_{T^{\dagger}_#2}$]{#1}{#2}{#3}}
\newcommand{\dTds}[3][]{\dAmpD{#1}{#2}{#3}}
\newcommand{\dUd}[3][]{\dAmpD[$_{U^{\dagger}_#2}$]{#1}{#2}{#3}}
\newcommand{\dTdv}[3][]{
\tikzset{exoper_line/.style={exvac_line_save}}
\dAmpD{#1}{#2}{#3}
\tikzset{exoper_line/.style={exoper_line_save}} }

%Hamilton parts
\newcommand{\dF}[1]{\dHone[$_{F}$]{#1}}
\newcommand{\dFs}[1]{\dHone{#1}}
\newcommand{\dX}[1]{\dHone[$_{X}$]{#1}}

\newcommand{\dW}[2]{\dHtwo[$_{W}$]{#1}{#2}}
\newcommand{\dWs}[2]{\dHtwo{#1}{#2}}
\newcommand{\dXtwo}[2]{\dHtwo[$_{X}$]{#1}{#2}}

\newcommand{\dline}[3][]{
\ifthenelse{\isempty{#1}}
{\draw[ph_line](#2) -- (#3);}
{\draw[ph_line](#2) -- node[inner sep=0pt,minimum size=0pt,right = 0.5pt] {\footnotesize\slshape\sffamily #1} (#3);}
}
\newcommand{\dcurve}[3][]{
\ifthenelse{\isempty{#1}}
{\draw[ph_line](#2) to [bend right=30] (#3);}
{\draw[ph_line](#2) to [bend right=30] node[inner sep=0pt,minimum size=0pt,right = 0.5pt] {\footnotesize\slshape\sffamily #1} (#3);}
}
\newcommand{\dcurver}[3][]{
\ifthenelse{\isempty{#1}}
{\draw[ph_line](#2) to [bend left=30] (#3);}
{\draw[ph_line](#2) to [bend left=30] node[inner sep=0pt,minimum size=0pt,right = 0.5pt] {\footnotesize\slshape\sffamily #1} (#3);}
}
\newcommand{\dcurcur}[2]{\dcurve{#1}{#2} \dcurve{#2}{#1}}
\newcommand{\dmovex}[1]{\FPeval\xcoorH{clip(\xcoord+#1*0.25*\sc)}}
\newcommand{\dmoveT}[1]{\FPeval\xcoor{clip(\xcoord+#1*0.25*\sc)}}
\newcommand{\dmoveTd}[1]{\FPeval\xcoordg{clip(\xcoord+#1*0.25*\sc)}}
\newcommand{\dmovac}[1]{\FPeval\yvac{clip(\yvac+#1*0.25*\sc)}}
\newcommand{\dname}[1]{
\FPeval\xx{clip(\xcoord+1*\sc)} \FPeval\result{clip(\yvac+0.25*\sc)} 
\node at (\xx,\result) {#1};}
\newcommand{\dtext}[2]{\FPeval\xx{clip(\xcoord+#1*\sc)} \FPeval\result{clip((\ycoord+\yvac)/2)} 
\node at (\xx,\result) {#2};}

\newcommand{\dscale}[1]{\FPset\sc{#1} 
\FPeval\xcoord{clip(-1*\sc)} \FPeval\ycoord{clip(-1*\sc)} 
\FPset\xcoor{\xcoord} \FPeval\ycoor{clip(\ycoord+1*\sc)} 
\FPset\xcoordg{\xcoord}
\FPset\xcoorH{\xcoord}
\FPeval\yvac{clip(\yvac*\sc)} \FPeval\xvac{clip(\xvac*\sc)}}
% end of Diagram-package


% test:
% \bdiag
% \dmovex{2}
% \dT{1}{t}
% \dT{2}{t2}
% \dW{wn1}{wn2}
% \dline{t1}{wn1}
% \dline{t1v1}{t1}
% \dline{t21}{wn2}
% \dline{t21}{t2v1}
% \dline{wn1}{wn1v2}
% \dline{wn2}{wn2v2}
% \dline{t22v1}{t22}
% \dline{t22}{t22v2}
% \ediag
