%
% \usepackage{tikz}
\usetikzlibrary{calc}
\usetikzlibrary{decorations.pathreplacing,decorations.markings}

%from tex.stackexchange.com
\tikzset{
  % style to apply some styles to each segment of a path
  on each segment/.style={
    decorate,
    decoration={
      show path construction,
      moveto code={},
      lineto code={
        \path [#1]
        (\tikzinputsegmentfirst) -- (\tikzinputsegmentlast);
      },
      curveto code={
        \path [#1] (\tikzinputsegmentfirst)
        .. controls
        (\tikzinputsegmentsupporta) and (\tikzinputsegmentsupportb)
        ..
        (\tikzinputsegmentlast);
      },
      closepath code={
        \path [#1]
        (\tikzinputsegmentfirst) -- (\tikzinputsegmentlast);
      },
    },
  },
  % style to add an arrow in the middle of a path
  mid arrow/.style={postaction={decorate,decoration={
        markings,
  %                     place for arrows
        mark=at position .62 with {\arrow[#1]{>}}
      }}},
  % style to add an arrow in the middle of a path
  mid rarrow/.style={postaction={decorate,decoration={
        markings,
  %                     place for arrows
        mark=at position .38 with {\arrow[#1]{<}}
      }}},
  % style to add a double-headed arrow in the middle of a path
  mid darrow/.style={postaction={decorate,decoration={
        markings,
  %                     place for arrows
        mark=at position .62 with {\arrow[#1]{>>}}
      }}},
  % style to add a double-headed arrow in the middle of a path
  mid rdarrow/.style={postaction={decorate,decoration={
        markings,
  %                     place for arrows
        mark=at position .38 with {\arrow[#1]{<<}}
      }}},
}

% from http://tex.stackexchange.com/questions/25678/nicer-wavy-line-with-tikz
\pgfdeclaredecoration{complete sines}{initial}
{
  \state{initial}[
        width=+0pt,
        next state=sine,
        persistent precomputation={\pgfmathsetmacro\matchinglength{
            \pgfdecoratedinputsegmentlength / int(\pgfdecoratedinputsegmentlength/\pgfdecorationsegmentlength)}
            \setlength{\pgfdecorationsegmentlength}{\matchinglength pt}
        }] {}
  \state{sine}[width=\pgfdecorationsegmentlength]{
        \pgfpathsine{\pgfpoint{0.25\pgfdecorationsegmentlength}{0.5\pgfdecorationsegmentamplitude}}
        \pgfpathcosine{\pgfpoint{0.25\pgfdecorationsegmentlength}{-0.5\pgfdecorationsegmentamplitude}}
        \pgfpathsine{\pgfpoint{0.25\pgfdecorationsegmentlength}{-0.5\pgfdecorationsegmentamplitude}}
        \pgfpathcosine{\pgfpoint{0.25\pgfdecorationsegmentlength}{0.5\pgfdecorationsegmentamplitude}}
  }
  \state{final}{}
}

%
%             CCDiag v.1.0 
%           D.Kats, September 2011
%
% \bdiag[<scale>] start diagram. If <scale> is given, the diagram will be scaled
% \bdiags[<scale>]: the H diagram will be placed in the middle (otherwise it is shifted halfway to the vacuum)
%
% \dT[<order>]{<exc.level>}{<node>} -> T_<exc.level>^{(<order>)}
% node-names are generated as <node>1, <node>2, <node>3, ...
% can be used without <order> (\dT{<exc.level>}{<node>})
% same with \dU (will print U_<exc.level> as label)
% for excitation operators without label use \dTs
% One can customize labels (see e.g. how \dU is defined)
% \dTd, \dTds, \dUd are T^{\dagger}, {}^{\dagger}, U^{\dagger}
%
% \dTdv[<order>]{<exc.level>}{<node>} : draw vacuum explicitly (\tau^\dagger) 
% \dTv[<order>]{<exc.level>}{<node>} -> \tau
%
% \dF{<node>} -> F
% \dFs{<node>} : one-electron operator without label
% \dX{<node>} : one-electron perturbation (X as label)
% \dHone[<name>]{<node>} : one-electron operator with label <name> 
% \dW{<left node>}{<right node>} -> W
% \dWs{<left node>}{<right node>} : two-electron operator without label
% \dXtwo{<left node>}{<right node>} : two-electron perturbation (X as label)
% \dHtwo[<name>]{<left node>}{<right node>} : two-electron operator with label <name>
%
% \dline[<index>]{<from node>}{<to node>} ->    "--->" (if <index> given - write <index> to the right of the line)
% \dcurve[<index>]{<from node>}{<to node>} ->   curved "--->" 
% \dcurver[<index>]{<from node>}{<to node>} ->   curved "--->" (reverse bend)
% \dcurcur{<node1>}{<node2>} ->     cycled curved "--->"
% \dcurt{<from node>}{<through node>}{<to node>} -> curved "--->" over three nodes! 
% \dcurtr{<from node>}{<through node>}{<to node>} ->   curved "--->" over three nodes (reverse bend) (not needed anymore!)
%
%  left vacuum-node for <node1> is called <node1>v1
%  right -------------"----------------   <node1>v2
%
% \dmovex{<value>} or \dmoveH{<value>} -> move W or F horizontally 
% \dmoveT{<value>} -> move T horizontally
% \dmovac{<value>} -> move vacuum (and daggers) vertically 
% \dmoveTd{<value>} -> move T^\dagger horizontally
%
% \dscale{<value>} -> scale size of diagrams with <value>
%
% \dname{<text>} -> write <text> over the diagram
% \dtext{<shift>}{<text>} -> write <text> in the diagram with horizontal shift
%
% change exoper-line-save, exvac-line-save, hoper-line-save, ph-line-save
%  in order to change excitation operator, explicit vacuum, H-operator, or p/h line styles.


\edef\xcoord{11} \edef\xcoor{11} \edef\xcoorbeg{11} \edef\xcoorend{11} \edef\result{11} \edef\ycoor{11} \edef\ycoord{11} %\edef\scal{11} 
\edef\heffsize{6}
\pgfmathsetmacro{\scalh}{1} %default scale-value
\pgfmathsetmacro{\scalv}{1} %default scale-value
\pgfmathsetmacro{\xcoord}{(-1*\scalh)} \pgfmathsetmacro{\ycoord}{(-1*\scalv)} 
\pgfmathsetmacro{\xcoor}{\xcoord} \pgfmathsetmacro{\ycoor}{(\ycoord+1*\scalv)} 
\pgfmathsetmacro{\xcoordg}{\xcoord}
\pgfmathsetmacro{\xcoorH}{\xcoord}
\pgfmathsetmacro{\yvac}{(\ycoor+0.5*\scalv)} \pgfmathsetmacro{\xvac}{(0.25*\scalh)} \pgfmathsetmacro{\xdvac}{(0.5*\scalh)}
\newcounter{stoer} 
\def\firstargum{}
\def\emptyargum{}
%p/h line styles
%arrows at the end 
%\tikzset{ph-line-arrow-save/.style={->,>=stealth,thin}}
%\tikzset{ph-liner-arrow-save/.style={ph-line-save,<-}}
%\tikzset{ph-line-darrow-save/.style={->>,>=stealth,thin}}
%\tikzset{ph-liner-darrow-save/.style={ph-line-save,<<-}}
%arrows in the middle
\tikzset{ph-line-arrow-save/.style={>=stealth,thin,postaction={on each segment={mid arrow}}}}
\tikzset{ph-liner-arrow-save/.style={>=stealth,thin,postaction={on each segment={mid rarrow}}}}
\tikzset{ph-line-darrow-save/.style={>=stealth,thin,postaction={on each segment={mid darrow}}}}
\tikzset{ph-liner-darrow-save/.style={>=stealth,thin,postaction={on each segment={mid rdarrow}}}}
\tikzset{ph-line-noarrow-save/.style={thin}}
\tikzset{ph-line-save/.style={ph-line-arrow-save}}
\tikzset{ph-liner-save/.style={ph-liner-arrow-save}}
%line styles
\tikzset{exoper-line-save/.style={very thick, solid}}
\tikzset{exvac-line-save/.style={thick, dotted}}
\tikzset{hoper-line-fey-save/.style={ decoration={complete sines,segment length=0.1cm,amplitude=0.1cm}, decorate}}
\tikzset{hoper-line-save/.style={very thick, dashed}}
\tikzset{hoperbar-line-save/.style={ hoper-line-save, double}}
\tikzset{heffoper-line-save/.style={thick, solid, double distance = \heffsize, line cap=rect }}
\tikzset{exoper2-line-save/.style={ thick, solid, double}}
\tikzset{exoper3-line-save/.style={ thick, transparent}}
\tikzset{exoper-line/.style={exoper-line-save}}
\tikzset{hoper-line/.style={hoper-line-save}}
\tikzset{heffoper-line/.style={heffoper-line-save}}
\tikzset{ph-line/.style={ph-line-save}}

\newcommand{\bdiag}[1][]{
 \ifx&#1&%
  \begin{tikzpicture}
 \else
  \begin{tikzpicture}[scale=#1]
 \fi
}
\newcommand{\bdiags}[1][]{
 \bdiag[#1]
  %symmetric diagram
   \pgfmathsetmacro{\yvac}{(\ycoor+1*\scalv)}
   \pgfmathsetmacro{\xvac}{(0.5*\scalh)}
}
\newcommand{\bdiagd}[1][]{
 %dagger diagram: ycoor moved to the bottom
 \pgfmathsetmacro{\ycoor}{(\ycoord+0.5*\scalv)}
 \pgfmathsetmacro{\xcoordg}{\xcoord}
 \pgfmathsetmacro{\xcoorH}{\xcoord}
 \pgfmathsetmacro{\yvac}{(\ycoor+1*\scalv)} \pgfmathsetmacro{\xvac}{(0.5*\scalh)} \pgfmathsetmacro{\xdvac}{(0.25*\scalh)}
 \bdiag[#1]
}

\newcommand{\ediag}{\end{tikzpicture}}
\newcommand{\ediags}{\ediag}
\newcommand{\ediagd}{\ediag}

\newcommand{\dorigin}[1][]{
 \ifx&#1&%
  \pgfmathsetmacro{\xorig}{0}
 \else
  \pgfmathsetmacro{\xorig}{#1}
 \fi
 \node(Orig) at (\xorig,0){};
}

\newcommand{\dAmp}[4][]{%for all
\pgfmathsetmacro{\xcoorbeg}{(\xcoor+0.5*\scalh)} 
\pgfmathsetmacro{\xcoorend}{(\xcoorbeg+#3*\scalh/2)} 
\draw[exoper-line](\xcoorbeg,\ycoord) -- (\xcoorend,\ycoord);
\pgfmathsetmacro{\result}{(\xcoorend+0.35*\scalh)} 
% write label if given
\ifx&#1&%
  \pgfmathsetmacro{\xcoor}{\xcoorend}
\else
  \node at (\result,\ycoord){#1};
  \pgfmathsetmacro{\xcoor}{(\xcoorend+0.25*\scalh)}
\fi
% calculate the vertex-distance (singles are a special case)
\ifnum#3=1
 \pgfmathsetmacro{\xxx}{((\xcoorend-\xcoorbeg)/2)} 
 \pgfmathsetmacro{\result}{\xcoorbeg}
\else
 \pgfmathsetmacro{\xxx}{((\xcoorend-\xcoorbeg)/( #3 - 1 ))} 
 \pgfmathsetmacro{\result}{(\xcoorbeg-\xxx)}
\fi
% set all nodes (and give them names)
\foreach \x in {1,...,#3}
{
  \coordinate (vertex) at ($(\result,\ycoord)+(\x*\xxx,0)$);
  \node[inner sep=0pt,minimum size=0pt] (#4\x) at  (vertex) {};
  \coordinate (vtxvac) at ($(\result,\yvac)+(\x*\xxx,0)$);
  \node[inner sep=0pt,minimum size=0pt] (#4\x v1) at ($(vtxvac)-(\xvac,0)$) {};
  \node[inner sep=0pt,minimum size=0pt] (#4\x v2) at ($(vtxvac)+(\xvac,0)$) {};
}
\node[inner sep=0pt,minimum size=0pt] (#4) at  (#41) {};
\node[inner sep=0pt,minimum size=0pt] (#4v1) at (#41v1) {};
\node[inner sep=0pt,minimum size=0pt] (#4v2) at (#41v2) {};
% set perturbation (if given)
\ifx&#2&%
%
\else  
\pgfmathsetmacro{\result}{((\xcoorend+\xcoorbeg)/2)} 
\setcounter{stoer}{#2} 
\node[inner sep=0pt,minimum size=0pt] at (\result,\ycoord){{\footnotesize\slshape\sffamily \Roman{stoer}}}; 
\fi
}

\newcommand{\dAmpD}[4][]{%for all
\pgfmathsetmacro{\xcoorbeg}{(\xcoordg+0.5*\scalh)} 
\pgfmathsetmacro{\xcoorend}{(\xcoorbeg+#3*\scalh/2)} 
\draw[exoper-line](\xcoorbeg,\yvac) -- (\xcoorend,\yvac);
\pgfmathsetmacro{\result}{(\xcoorend+0.35*\scalh)} 
% write label if given
\ifx&#1&%
\pgfmathsetmacro{\xcoordg}{\xcoorend}
\else
\node at (\result,\yvac){#1};
\pgfmathsetmacro{\xcoordg}{(\xcoorend+0.25*\scalh)}
\fi
% calculate the vertex-distance (singles are a special case)
\ifnum#3=1
 \pgfmathsetmacro{\xxx}{((\xcoorend-\xcoorbeg)/2)} 
 \pgfmathsetmacro{\result}{\xcoorbeg}
\else
 \pgfmathsetmacro{\xxx}{((\xcoorend-\xcoorbeg)/( #3 - 1 ))} 
 \pgfmathsetmacro{\result}{(\xcoorbeg-\xxx)}
\fi
% set all nodes (and give them names)
\foreach \x in {1,...,#3}
{
  \coordinate (vertex) at ($(\result,\yvac)+(\x*\xxx,0)$);
  \node[inner sep=0pt,minimum size=0pt] (#4\x) at  (vertex) {};
  \coordinate (vtxvac) at ($(\result,\ycoord)+(\x*\xxx,0)$);
  \node[inner sep=0pt,minimum size=0pt] (#4\x v1) at ($(vtxvac)-(\xdvac,0)$) {};
  \node[inner sep=0pt,minimum size=0pt] (#4\x v2) at ($(vtxvac)+(\xdvac,0)$) {};
}
\node[inner sep=0pt,minimum size=0pt] (#4) at  (#41) {};
\node[inner sep=0pt,minimum size=0pt] (#4v1) at (#41v1) {};
\node[inner sep=0pt,minimum size=0pt] (#4v2) at (#41v2) {};
% set perturbation (if given)
\ifx&#2&%
%
\else
\pgfmathsetmacro{\result}{((\xcoorend+\xcoorbeg)/2)} 
\setcounter{stoer}{#2} 
\node[inner sep=0pt,minimum size=0pt] at (\result,\yvac){{\footnotesize\slshape\sffamily \Roman{stoer}}}; 
\fi
}

\newcommand{\dHeff}[4][]{%for all
\pgfmathsetmacro{\xcoorbeg}{(\xcoorH+0.5*\scalh)} 
\pgfmathsetmacro{\xcoorend}{(\xcoorbeg+#3*\scalh/2)} 
\draw[heffoper-line](\xcoorbeg,\ycoor) -- (\xcoorend,\ycoor);
\pgfmathsetmacro{\ycoora}{(\ycoor+0.018*\heffsize)} 
\pgfmathsetmacro{\ycoorb}{(\ycoor-0.018*\heffsize)} 
\pgfmathsetmacro{\result}{(\xcoorend+0.6*\scalh)} 
% write label if given
\ifx&#1&%
  \pgfmathsetmacro{\xcoorH}{\xcoorend}
\else
  \node at (\result,\ycoor){#1};
  \pgfmathsetmacro{\xcoorH}{(\xcoorend+0.25*\scalh)}
\fi
% calculate the vertex-distance (singles are a special case)
\ifnum#3=1
 \pgfmathsetmacro{\xxx}{((\xcoorend-\xcoorbeg)/2)} 
 \pgfmathsetmacro{\result}{\xcoorbeg}
\else
 \pgfmathsetmacro{\xxx}{((\xcoorend-\xcoorbeg)/( #3 - 1 ))} 
 \pgfmathsetmacro{\result}{(\xcoorbeg-\xxx)}
\fi
% set all nodes (and give them names)
\foreach \x in {1,...,#3}
{
  \coordinate (vertex) at ($(\result,\ycoora)+(\x*\xxx,0)$);
  \node[inner sep=0pt,minimum size=0pt] (#4\x a) at  (vertex) {};
  \coordinate (vertex) at ($(\result,\ycoorb)+(\x*\xxx,0)$);
  \node[inner sep=0pt,minimum size=0pt] (#4\x b) at  (vertex) {};
  \coordinate (vtxvac) at ($(\result,\yvac)+(\x*\xxx,0)$);
  \node[inner sep=0pt,minimum size=0pt] (#4\x v1) at ($(vtxvac)-(\xvac,0)$) {};
  \node[inner sep=0pt,minimum size=0pt] (#4\x v2) at ($(vtxvac)+(\xvac,0)$) {};
  \coordinate (vtxvac) at ($(\result,\ycoord)+(\x*\xxx,0)$);
  \node[inner sep=0pt,minimum size=0pt] (#4\x vd1) at ($(vtxvac)-(\xvac,0)$) {};
  \node[inner sep=0pt,minimum size=0pt] (#4\x vd2) at ($(vtxvac)+(\xvac,0)$) {};
}
\node[inner sep=0pt,minimum size=0pt] (#4a) at  (#41a) {};
\node[inner sep=0pt,minimum size=0pt] (#4b) at  (#41b) {};
\node[inner sep=0pt,minimum size=0pt] (#4v1) at (#41v1) {};
\node[inner sep=0pt,minimum size=0pt] (#4v2) at (#41v2) {};
\node[inner sep=0pt,minimum size=0pt] (#4vd1) at (#41vd1) {};
\node[inner sep=0pt,minimum size=0pt] (#4vd2) at (#41vd2) {};
% set perturbation (if given)
\ifx&#2&%
%
\else  
\pgfmathsetmacro{\result}{((\xcoorend+\xcoorbeg)/2)} 
\setcounter{stoer}{#2} 
\node[inner sep=0pt,minimum size=0pt] at (\result,\ycoor){{\footnotesize\slshape\sffamily \Roman{stoer}}}; 
\fi
}


\newcommand{\dCross}{$\mathbin{\tikz [x=1.4ex,y=1.4ex] \draw[thick] (0,0) -- (1,1) (0,1) -- (1,0);}$}

\newcommand{\dHone}[2][]{\pgfmathsetmacro{\xcoorbeg}{(\xcoorH+0.5*\scalh)}
\pgfmathsetmacro{\xcoorend}{(\xcoorbeg+0.5*\scalh)} 
\draw[hoper-line](\xcoorbeg,\ycoor) -- (\xcoorend,\ycoor);
%\node at (\xcoorend,\ycoor){$\bf \times$};
\node at (\xcoorend,\ycoor){\dCross};
\pgfmathsetmacro{\result}{(\xcoorend+0.35*\scalh)} 
%name of operator
\ifx&#1&%
\pgfmathsetmacro{\xcoorH}{\xcoorend}
\else
\node at (\result,\ycoor){#1};
\pgfmathsetmacro{\xcoorH}{(\xcoorend+0.25*\scalh)}
\fi
\node[inner sep=0pt,minimum size=0pt] (#2) at (\xcoorbeg,\ycoor){}; 
\pgfmathsetmacro{\xx}{(\xcoorbeg-\xvac/2)} 
\node[inner sep=0pt,minimum size=0pt] (#2v1) at (\xx,\yvac) {}; 
\pgfmathsetmacro{\xx}{(\xcoorbeg+\xvac/2)} 
\node[inner sep=0pt,minimum size=0pt] (#2v2) at (\xx,\yvac) {};
\pgfmathsetmacro{\xx}{(\xcoorbeg-\xdvac/2)}
\node[inner sep=0pt,minimum size=0pt] (#2vd1) at (\xx,\ycoord) {};
\pgfmathsetmacro{\xx}{(\xcoorbeg+\xdvac/2)}
\node[inner sep=0pt,minimum size=0pt] (#2vd2) at (\xx,\ycoord) {};
\node[inner sep=0pt,minimum size=0pt] (#21) at  (#2) {};
\node[inner sep=0pt,minimum size=0pt] (#21v1) at (#2v1) {};
\node[inner sep=0pt,minimum size=0pt] (#21v2) at (#2v2) {};
}

\newcommand{\dHoner}[2][]{\pgfmathsetmacro{\xcoorbeg}{(\xcoorH+0.5*\scalh)}
\pgfmathsetmacro{\xcoorend}{(\xcoorbeg+0.5*\scalh)} 
\draw[hoper-line](\xcoorbeg,\ycoor) -- (\xcoorend,\ycoor);
\node at (\xcoorbeg,\ycoor){$\bf \times$};
\pgfmathsetmacro{\result}{(\xcoorend+0.35*\scalh)} 
%name of operator
\ifx&#1&%
\pgfmathsetmacro{\xcoorH}{\xcoorend}
\else
\node at (\result,\ycoor){#1};
\pgfmathsetmacro{\xcoorH}{(\xcoorend+0.25*\scalh)}
\fi
\node[inner sep=0pt,minimum size=0pt] (#2) at (\xcoorend,\ycoor){}; 
\pgfmathsetmacro{\xx}{(\xcoorend-\xvac/2)} 
\node[inner sep=0pt,minimum size=0pt] (#2v1) at (\xx,\yvac) {}; 
\pgfmathsetmacro{\xx}{(\xcoorend+\xvac/2)} 
\node[inner sep=0pt,minimum size=0pt] (#2v2) at (\xx,\yvac) {};
\pgfmathsetmacro{\xx}{(\xcoorend-\xdvac/2)}
\node[inner sep=0pt,minimum size=0pt] (#2vd1) at (\xx,\ycoord) {};
\pgfmathsetmacro{\xx}{(\xcoorend+\xdvac/2)}
\node[inner sep=0pt,minimum size=0pt] (#2vd2) at (\xx,\ycoord) {};
}

\newcommand{\dHtwo}[3][]{ 
\pgfmathsetmacro{\xcoorbeg}{(\xcoorH+0.5*\scalh)}
\pgfmathsetmacro{\xcoorend}{(\xcoorbeg+1*\scalh)} 
\draw[hoper-line](\xcoorbeg,\ycoor) -- (\xcoorend,\ycoor); 
\pgfmathsetmacro{\result}{(\xcoorend+0.35*\scalh)} 
%name of operator
\ifx&#1&%
\pgfmathsetmacro{\xcoorH}{\xcoorend}
\else
\node at (\result,\ycoor){#1};
\pgfmathsetmacro{\xcoorH}{(\xcoorend+0.25*\scalh)}
\fi
\node[inner sep=0pt,minimum size=0pt] (#2) at (\xcoorbeg,\ycoor) {};
\node[inner sep=0pt,minimum size=0pt] (#3) at (\xcoorend,\ycoor) {};
\pgfmathsetmacro{\xx}{(\xcoorbeg-\xvac/2)} 
\node[inner sep=0pt,minimum size=0pt] (#2v1) at (\xx,\yvac) {};
\pgfmathsetmacro{\xx}{(\xcoorbeg+\xvac/2)} 
\node[inner sep=0pt,minimum size=0pt] (#2v2) at (\xx,\yvac) {}; 
\pgfmathsetmacro{\xx}{(\xcoorend-\xvac/2)} 
\node[inner sep=0pt,minimum size=0pt] (#3v1) at (\xx,\yvac) {};
\pgfmathsetmacro{\xx}{(\xcoorend+\xvac/2)} 
\node[inner sep=0pt,minimum size=0pt] (#3v2) at (\xx,\yvac) {};
\pgfmathsetmacro{\xx}{(\xcoorbeg-\xdvac/2)} 
\node[inner sep=0pt,minimum size=0pt] (#2vd1) at (\xx,\ycoord) {};
\pgfmathsetmacro{\xx}{(\xcoorbeg+\xdvac/2)} 
\node[inner sep=0pt,minimum size=0pt] (#2vd2) at (\xx,\ycoord) {}; 
\pgfmathsetmacro{\xx}{(\xcoorend-\xdvac/2)} 
\node[inner sep=0pt,minimum size=0pt] (#3vd1) at (\xx,\ycoord) {};
\pgfmathsetmacro{\xx}{(\xcoorend+\xdvac/2)} 
\node[inner sep=0pt,minimum size=0pt] (#3vd2) at (\xx,\ycoord) {};
} 

%general excitations
\newcommand{\dT}[3][]{\dAmp[$_{T_#2}$]{#1}{#2}{#3}}
\newcommand{\dTs}[3][]{\dAmp{#1}{#2}{#3}}

\newcommand{\dU}[3][]{
\tikzset{exoper-line/.style={exoper2-line-save}}
\dAmp[$_{U_#2}$]{#1}{#2}{#3}
\tikzset{exoper-line/.style={exoper-line-save}} }

\newcommand{\dUs}[3][]{
\tikzset{exoper-line/.style={exoper2-line-save}}
\dAmp{#1}{#2}{#3}
\tikzset{exoper-line/.style={exoper-line-save}} }

\newcommand{\dTt}[3][]{
\tikzset{exoper-line/.style={exoper3-line-save}}
\dAmp{#1}{#2}{#3}
\tikzset{exoper-line/.style={exoper-line-save}} }

\newcommand{\dTv}[3][]{
\tikzset{exoper-line/.style={exvac-line-save}}
\dAmp{#1}{#2}{#3}
\tikzset{exoper-line/.style={exoper-line-save}} }

\newcommand{\dTd}[3][]{\dAmpD[$_{T^{\dagger}_#2}$]{#1}{#2}{#3}}
\newcommand{\dTds}[3][]{\dAmpD{#1}{#2}{#3}}

\newcommand{\dUd}[3][]{
\tikzset{exoper-line/.style={exoper2-line-save}}
\dAmpD[$_{U^{\dagger}_#2}$]{#1}{#2}{#3}
\tikzset{exoper-line/.style={exoper-line-save}} }

\newcommand{\dUds}[3][]{
\tikzset{exoper-line/.style={exoper2-line-save}}
\dAmpD{#1}{#2}{#3}
\tikzset{exoper-line/.style={exoper-line-save}} }

\newcommand{\dTtd}[3][]{
\tikzset{exoper-line/.style={exoper3-line-save}}
\dAmpD{#1}{#2}{#3}
\tikzset{exoper-line/.style={exoper-line-save}} }

\newcommand{\dTdv}[3][]{
\tikzset{exoper-line/.style={exvac-line-save}}
\dAmpD{#1}{#2}{#3}
\tikzset{exoper-line/.style={exoper-line-save}} }

%excitations with named nodes
\newcommand{\dTone}[3][]{\dAmp[$_{T_1}$]{#1}{1}{#2}; \draw (#2) node[below] {\footnotesize\slshape\sffamily #3};}
\newcommand{\dTtwo}[4][]{\dAmp[$_{T_2}$]{#1}{2}{#2}; \draw (#21) node[below] {\footnotesize\slshape\sffamily #3};\draw (#22) node[below] {\footnotesize\slshape\sffamily #4};}
\newcommand{\dTsone}[3][]{\dAmp{#1}{1}{#2}; \draw (#2) node[below] {\footnotesize\slshape\sffamily #3};}
\newcommand{\dTstwo}[4][]{\dAmp{#1}{2}{#2}; \draw (#21) node[below] {\footnotesize\slshape\sffamily #3};\draw (#22) node[below] {\footnotesize\slshape\sffamily #4};}
\newcommand{\dTdone}[3][]{\dAmpD[$_{T^{\dagger}_1}$]{#1}{1}{#2}; \draw (#2) node[above] {\footnotesize\slshape\sffamily #3};}
\newcommand{\dTdtwo}[4][]{\dAmpD[$_{T^{\dagger}_2}$]{#1}{2}{#2}; \draw (#21) node[above] {\footnotesize\slshape\sffamily #3};\draw (#22) node[above] {\footnotesize\slshape\sffamily #4};}
\newcommand{\dTdsone}[3][]{\dAmpD{#1}{1}{#2}; \draw (#2) node[above] {\footnotesize\slshape\sffamily #3};}
\newcommand{\dTdstwo}[4][]{\dAmpD{#1}{2}{#2}; \draw (#21) node[above] {\footnotesize\slshape\sffamily #3};\draw (#22) node[above] {\footnotesize\slshape\sffamily #4};}
\newcommand{\dTdvone}[3][]{
  \tikzset{exoper-line/.style={exvac-line-save}}
  % set perturbation (if given)
  \ifx&#1&%
    \dTdsone{#2}{#3}
  \else
    \dTdsone[#1]{#2}{#3}
  \fi
  \tikzset{exoper-line/.style={exoper-line-save}} }
\newcommand{\dTdvtwo}[4][]{
  \tikzset{exoper-line/.style={exvac-line-save}}
  % set perturbation (if given)
  \ifx&#1&%
    \dTdstwo{#2}{#3}{#4}
  \else
    \dTdstwo[#1]{#2}{#3}{#4}
  \fi
  \tikzset{exoper-line/.style={exoper-line-save}} }
\newcommand{\dUone}[3][]{
  \tikzset{exoper-line/.style={exoper2-line-save}}
  % set perturbation (if given)
  \ifx&#1&%
    \dTone{#2}{#3}
  \else
    \dTone[#1]{#2}{#3}
  \fi
  \tikzset{exoper-line/.style={exoper-line-save}} 
}
\newcommand{\dUtwo}[4][]{
  \tikzset{exoper-line/.style={exoper2-line-save}}
  % set perturbation (if given)
  \ifx&#1&%
    \dTtwo{#2}{#3}{#4}
  \else
    \dTtwo[#1]{#2}{#3}{#4}
  \fi
  \tikzset{exoper-line/.style={exoper-line-save}} 
}
\newcommand{\dUsone}[3][]{
  \tikzset{exoper-line/.style={exoper2-line-save}}
  % set perturbation (if given)
  \ifx&#1&%
    \dTsone{#2}{#3}
  \else
    \dTsone[#1]{#2}{#3}
  \fi
  \tikzset{exoper-line/.style={exoper-line-save}} 
}
\newcommand{\dUstwo}[4][]{
  \tikzset{exoper-line/.style={exoper2-line-save}}
  % set perturbation (if given)
  \ifx&#1&%
    \dTstwo{#2}{#3}{#4}
  \else
    \dTstwo[#1]{#2}{#3}{#4}
  \fi
  \tikzset{exoper-line/.style={exoper-line-save}} 
}


%Hamilton parts
\newcommand{\dHeffs}[3][]{\dHeff{#1}{#2}{#3}}

\newcommand{\dF}[1]{\dHone[$_{F}$]{#1}}
\newcommand{\dFs}[1]{\dHone{#1}}
\newcommand{\dFbar}[1]{
  \tikzset{hoper-line/.style={hoperbar-line-save}}
  \dHone[$_{F}$]{#1}
  \tikzset{hoper-line/.style={hoper-line-save}}
}
\newcommand{\dFbars}[1]{
  \tikzset{hoper-line/.style={hoperbar-line-save}}
  \dHone{#1}
  \tikzset{hoper-line/.style={hoper-line-save}}
}
\newcommand{\dX}[1]{\dHone[$_{X}$]{#1}}

\newcommand{\dFr}[1]{\dHoner[$_{F}$]{#1}}
\newcommand{\dFsr}[1]{\dHoner{#1}}
\newcommand{\dXr}[1]{\dHoner[$_{X}$]{#1}}

\newcommand{\dW}[2]{\dHtwo[$_{W}$]{#1}{#2}}
\newcommand{\dWs}[2]{\dHtwo{#1}{#2}}
\newcommand{\dWbar}[2]{
  \tikzset{hoper-line/.style={hoperbar-line-save}}
  \dHtwo[$_{\bar W}$]{#1}{#2}
  \tikzset{hoper-line/.style={hoper-line-save}}
}
\newcommand{\dWbars}[2]{
  \tikzset{hoper-line/.style={hoperbar-line-save}}
  \dHtwo{#1}{#2}
  \tikzset{hoper-line/.style={hoper-line-save}}
}
\newcommand{\dXtwo}[2]{\dHtwo[$_{X}$]{#1}{#2}}
%Hamilton parts with named nodes
\newcommand{\dFn}[2]{\dHone[$_{F}$]{#1}; \draw (#1) node[below] {\footnotesize\slshape\sffamily #2};}
\newcommand{\dFsn}[2]{\dHone{#1}; \draw (#1) node[below] {\footnotesize\slshape\sffamily #2};}
\newcommand{\dFrn}[2]{\dHoner[$_{F}$]{#1}; \draw (#1) node[below] {\footnotesize\slshape\sffamily #2};}
\newcommand{\dFsrn}[2]{\dHoner{#1}; \draw (#1) node[below] {\footnotesize\slshape\sffamily #2};}
\newcommand{\dWn}[4]{\dHtwo[$_{W}$]{#1}{#2}; \draw (#1) node[above] {\footnotesize\slshape\sffamily #3};\draw (#2) node[above] {\footnotesize\slshape\sffamily #4};}
\newcommand{\dWsn}[4]{\dHtwo{#1}{#2}; \draw (#1) node[above] {\footnotesize\slshape\sffamily #3};\draw (#2) node[above] {\footnotesize\slshape\sffamily #4};}



\newcommand{\dline}[3][]{
\ifx&#1&%
\draw[ph-line](#2) -- (#3);
\else
\draw[ph-line](#2) -- node[inner sep=0pt,minimum size=0pt,right = 0.5pt] {\footnotesize\slshape\sffamily #1} (#3);
\fi
}
\newcommand{\dcurve}[3][]{
\ifx&#1&%
\draw[ph-line](#2) to [bend right=30] (#3);
\else
\draw[ph-line](#2) to [bend right=30] node[inner sep=0pt,minimum size=0pt,right = 0.5pt] {\footnotesize\slshape\sffamily #1} (#3);
\fi
}
\newcommand{\dcurver}[3][]{
  \tikzset{ph-line/.style={ph-liner-save}}
  \dcurve[#1]{#3}{#2}
  \tikzset{ph-line/.style={ph-line-save}}
}
\newcommand{\dcurcur}[2]{\dcurve{#1}{#2} \dcurve{#2}{#1}}

\newcommand{\dcurt}[3]{
  \draw[ph-line] let \p1 = (#1), \p2 = (#2)
    in \pgfextra{
        \pgfmathsetmacro{\dx}{\x2==\x1}
        \pgfmathsetmacro{\dy}{\y2>\y1}
        \pgfmathsetmacro{\ang}{90-2*atan((\y2-\y1)/(\x2-\x1+\dx/10))}
        \pgfmathsetmacro{\ango}{180*(1-\dy) - \ang +1}
        \pgfmathsetmacro{\angi}{90-180*\dy +1}
       }
    (#1) to [out=\ango,in=\angi] (#2);

  \draw[ph-line] let \p1 = (#2), \p2 = (#3)
    in \pgfextra{
        \pgfmathsetmacro{\dx}{\x2==\x1}
        \pgfmathsetmacro{\dy}{\y2>\y1}
        \pgfmathsetmacro{\ang}{90-2*atan((\y2-\y1)/(\x2-\x1+\dx/10))}
        \pgfmathsetmacro{\angi}{-180*\dy - \ang + 361}
        \pgfmathsetmacro{\ango}{180*\dy -89 }
       } 
    (#2) to [out=\ango,in=\angi] (#3);
}
% switch to this one if TexLive2011 is available
%\newcommand{\dcurt}[3]{
  %\draw[ph-line] let \p1 = (#1), \p2 = (#2)
    %in \pgfextra{
        %\pgfmathsetmacro{\dx}{\x2==\x1 ? 1 : \x2-\x1 } 
        %\pgfmathsetmacro{\ang}{\x2==\x1? -90 :90-2*atan((\y2-\y1)/(\dx))}
        %\pgfmathsetmacro{\sang}{\y2>\y1? -1 : 1}
        %\pgfmathsetmacro{\ango}{\y2>\y1? -\ang : 180 - \ang}
        %\pgfmathsetmacro{\angi}{\y2>\y1? -90 : 90}
       %}
    %(#1) to [out=\ango,in=\angi] (#2);
  %\draw[ph-line] let \p1 = (#2), \p2 = (#3)
    %in \pgfextra{
        %\pgfmathsetmacro{\dx}{\x2==\x1 ? 1 : \x2-\x1 }
        %\pgfmathsetmacro{\ang}{\x2==\x1 ? -90 : 90+2*atan((\y2-\y1)/(\dx))}
        %\pgfmathsetmacro{\angi}{\y2>\y1? \ang : 180 + \ang}
        %\pgfmathsetmacro{\ango}{\y2>\y1? 90 : -90}
       %}
    %(#2) to [out=\ango,in=\angi] (#3);
%}
\newcommand{\dcurtr}[3]{
  \tikzset{ph-line/.style={ph-liner-save}}
  \dcurt{#3}{#2}{#1}
  \tikzset{ph-line/.style={ph-line-save}}  
}

\newcommand{\dbubble}[2]{
  \draw[ph-line] (#1) arc (180:360:#2*0.25*\scalh) node (#1cir) {};
  \draw[ph-line] (#1cir) arc (0:180:#2*0.25*\scalh) {};
}
\newcommand{\dbubbler}[2]{
  \tikzset{ph-line/.style={ph-liner-save}}
  \dbubble{#1}{#2}
  \tikzset{ph-line/.style={ph-line-save}}  
}
\newcommand{\doyster}[2]{
  \draw[ph-line](#1) to [bend right=60] (#2);
}
\newcommand{\doysterr}[2]{
  \tikzset{ph-line/.style={ph-liner-save}}
  \doyster{#1}{#2}
  \tikzset{ph-line/.style={ph-line-save}}  
}

\newcommand{\dmoveH}[1]{\pgfmathsetmacro{\xcoorH}{(\xcoorH+#1*0.25*\scalh)}}
\newcommand{\dmoveT}[1]{\pgfmathsetmacro{\xcoor}{(\xcoor+#1*0.25*\scalh)}}
\newcommand{\dmoveTd}[1]{\pgfmathsetmacro{\xcoordg}{(\xcoordg+#1*0.25*\scalh)}}
\newcommand{\dmovac}[1]{\pgfmathsetmacro{\yvac}{(\yvac+#1*0.25*\scalv)}}
\newcommand{\dmovex}[1]{\dmoveH{#1}} %alias
\newcommand{\dvmoveH}[1]{\pgfmathsetmacro{\ycoor}{(\ycoor+#1*0.25*\scalv)}}
\newcommand{\dvmoveT}[1]{\pgfmathsetmacro{\ycoord}{(\ycoord+#1*0.25*\scalv)}}

\newcommand{\dname}[1]{
\pgfmathsetmacro{\xx}{(\xcoord+1*\scalh)} \pgfmathsetmacro{\result}{(\yvac+0.25*\scalv)} 
\node at (\xx,\result) {#1};}
\newcommand{\dtext}[2]{\pgfmathsetmacro{\xx}{(\xcoord+#1*\scalh)} \pgfmathsetmacro{\result}{((\ycoord+\yvac)/2)} 
\node at (\xx,\result) {#2};}

\newcommand{\dvscale}[1]{
  \pgfmathsetmacro{\scalv}{#1} 
  \pgfmathsetmacro{\ycoord}{(-1*\scalv)} 
  \pgfmathsetmacro{\ycoor}{(\ycoord+1*\scalv)} 
  \pgfmathsetmacro{\yvac}{(\yvac*\scalv)} 
}
\newcommand{\dhscale}[1]{
  \pgfmathsetmacro{\scalh}{#1} 
  \pgfmathsetmacro{\xcoord}{(-1*\scalh)} 
  \pgfmathsetmacro{\xcoor}{\xcoord} 
  \pgfmathsetmacro{\xcoordg}{\xcoord}
  \pgfmathsetmacro{\xcoorH}{\xcoord}
  \pgfmathsetmacro{\xvac}{(\xvac*\scalh)}
}

\newcommand{\dscale}[1]{
  \dvscale{#1}
  \dhscale{#1}
}

\newcommand{\dscaleop}[1]{\pgfmathsetmacro{\scalh}{#1}}

\newcommand{\dheffsize}[1]{\pgfmathsetmacro{\heffsize}{#1}}

\newcommand{\dfeynman}{
  \tikzset{hoper-line-save/.style={hoper-line-fey-save}}
  \tikzset{hoperbar-line-save/.style={hoper-line-save, double}}
}

\newcommand{\dnoarrow}{
\tikzset{ph-line-save/.style={ph-line-noarrow-save}}
\tikzset{ph-liner-save/.style={ph-line-noarrow-save}}
\tikzset{ph-line/.style={ph-line-save}}
}
\newcommand{\darrow}{
\tikzset{ph-line-save/.style={ph-line-arrow-save}}
\tikzset{ph-liner-save/.style={ph-liner-arrow-save}}
\tikzset{ph-line/.style={ph-line-save}}
}
\newcommand{\ddoubleheadarrow}{
\tikzset{ph-line-save/.style={ph-line-darrow-save}}
\tikzset{ph-liner-save/.style={ph-liner-darrow-save}}
\tikzset{ph-line/.style={ph-line-save}}
}

% end of CCDiag


% test:
% \bdiag
% \dmovex{2}
% \dT{1}{t}
% \dT{2}{t2}
% \dW{wn1}{wn2}
% \dline{t1}{wn1}
% \dline{t1v1}{t1}
% \dline{t21}{wn2}
% \dline{t21}{t2v1}
% \dline{wn1}{wn1v2}
% \dline{wn2}{wn2v2}
% \dline{t22v1}{t22}
% \dline{t22}{t22v2}
% \ediag
