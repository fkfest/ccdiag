\documentclass[a4paper]{article}
\usepackage{fullpage}
\setlength\parindent{0pt}

%ccdiag
\usepackage{tikz}
%
% \usepackage{tikz}
% \usepackage{xifthen}
% \usepackage{fp}
\usetikzlibrary{calc}
%
%       new commands (-> Diagram-package)
%           D.Kats, September 2011
%
% \bdiag[s] start diagram. If s is given, the H diagram is in the middle (otherwise it is shifted to the vacuum)
%
% \dT[<order>]{<exc.level>}{<node>} -> T_<exc.level>^{(<order>)}
% node-names are generated as <node>1, <node>2, <node>3, ...
% can be used without <order> (\dT{<exc.level>}{<node>})
% same with \dU (will print U_<exc.level> as label)
% for excitation operators without label use \dTs
% One can customize labels (see e.g. how \dU is defined)
% \dTd, \dTds, \dUd are T^{\dagger}, {}^{\dagger}, U^{\dagger}
%
% \dTdv[<order>]{<exc.level>}{<node>} : draw vacuum explicitly (\tau^\dagger) 
% \dTv[<order>]{<exc.level>}{<node>} -> \tau
%
% \dF{<node>} -> F
% \dFs{<node>} : one-electron operator without label
% \dX{<node>} : one-electron perturbation (X as label)
% \dHone[<name>]{<node>} : one-electron operator with label <name> 
% \dW{<left node>}{<right node>} -> W
% \dWs{<left node>}{<right node>} : two-electron operator without label
% \dXtwo{<left node>}{<right node>} : two-electron perturbation (X as label)
% \dHtwo[<name>]{<left node>}{<right node>} : two-electron operator with label <name>
%
% \dline[<index>]{<from node>}{<to node>} ->    "--->" (if <index> given - write <index> to the right of the line)
% \dcurve[<index>]{<from node>}{<to node>} ->   curved "--->" 
% \dcurver[<index>]{<from node>}{<to node>} ->   curved "--->" (reverse bend)
% \dcurcur{<node1>}{<node2>} ->     cycled curved "--->"
%  left vacuum-node for <node1> is called <node1>v1
%  right -------------"----------------   <node1>v2
%
% \dmovex{<value>} -> move W or F horizontally
% \dmoveT{<value>} -> move T horizontally
% \dmovac{<value>} -> move vacuum (and daggers) vertically 
% \dmoveTd{<value>} -> move T^\dagger horizontally
%
% \dscale{<value>} -> scale size of diagrams with <value>
%
% \dname{<text>} -> write <text> over the diagram
% \dtext{<text>} -> write <text> in the diagram
%
% change exoper_line_save, exvac_line_save, hoper_line_save, ph_line_save
%  in order to change excitation operator, explicit vacuum, H-operator, or p/h line styles.
%
% Old commands (useful for custom node-names):
% \dTone{<node>} -> T_1
% \dTtwo{<left node>}{<right node>} -> T_2
% \dTthr{<left node>}{<middle node>}{<right node>} -> T_3
%
% \dTone[<order>]{<node>} -> T_1^{(<order>)}
% \dTtwo[<order>]{<left node>}{<right node>} -> T_2^{(<order>)}
% \dTthr[<order>]{<left node>}{<middle node>}{<right node>} -> T_3^{(<order>)}
%
% same with \dUone, \dUtwo, \dUthr (will print U_1, U_2 or U_3)
% for excitation operators without label use \dTones, \dTtwos, \dTthrs
% One can customize labels (see e.g. how \dUone is defined)


\edef\xcoord{11} \edef\xcoor{11} \edef\xcoorbeg{11} \edef\xcoorend{11} \edef\result{11} \edef\ycoor{11} \edef\ycoord{11} \edef\sc{11} 
\FPset\sc{1} %default scale-value
\FPeval\xcoord{clip(-1*\sc)} \FPeval\ycoord{clip(-1*\sc)} 
\FPset\xcoor{\xcoord} \FPeval\ycoor{clip(\ycoord+1*\sc)} 
\FPset\xcoordg{\xcoord}
\FPset\xcoorH{\xcoord}
\FPeval\yvac{clip(\ycoor+0.5*\sc)} \FPeval\xvac{clip(0.25*\sc)}
\newcounter{stoer} 
\tikzset{exoper_line_save/.style={very thick, solid}}
\tikzset{exvac_line_save/.style={very thick, dotted}}
\tikzset{hoper_line_save/.style={very thick, dashed}}
\tikzset{ph_line_save/.style={->,>=stealth,thin}}
\tikzset{exoper_line/.style={exoper_line_save}}
\tikzset{hoper_line/.style={hoper_line_save}}
\tikzset{ph_line/.style={ph_line_save}}

\newcommand{\bdiag}[1][]{
  \begin{tikzpicture}
    \ifthenelse{\isempty{#1}}
      {}
      {%symmetric diagram
        \FPeval\yvac{clip(\ycoor+1*\sc)}
        \FPeval\xvac{clip(0.5*\sc)}
      }
}
\newcommand{\ediag}{\end{tikzpicture}}

\newcommand{\dAmpOne}[3][]{\FPeval\xcoorbeg{clip(\xcoor+0.5*\sc)} 
\FPeval\xcoorend{clip(\xcoorbeg+0.5*\sc)} 
\draw[very thick](\xcoorbeg,\ycoord) -- (\xcoorend,\ycoord);
\FPeval\result{clip(\xcoorend+0.35*\sc)} 
\ifthenelse{\isempty{#1}}  
{\FPeval\xcoor{\xcoorend}}
{\node at (\result,\ycoord){#1};
\FPeval\xcoor{clip(\xcoorend+0.25*\sc)}} 
\FPeval\result{clip((\xcoorbeg+\xcoorend)/2)} 
\node[inner sep=0pt,minimum size=0pt] (#3) at  (\result,\ycoord) {};
\FPeval\xx{clip(\xcoorbeg-\xvac)} 
\node[inner sep=0pt,minimum size=0pt] (#3v1) at (\xx,\yvac) {}; 
\FPeval\xx{clip(\xcoorbeg+\xvac)} 
\node[inner sep=0pt,minimum size=0pt] (#3v2) at (\xx,\yvac) {};
\ifthenelse{\isempty{#2}}  {}  
{ 
\FPeval\result{clip((\xcoorend+\xcoorbeg)/2)} 
\setcounter{stoer}{#2} 
\node[inner sep=0pt,minimum size=0pt] at (\result,\ycoord){{\footnotesize\slshape\sffamily \Roman{stoer}}}; }
}

\newcommand{\dAmpTwo}[4][]{\FPeval\xcoorbeg{clip(\xcoor+0.5*\sc)} 
\FPeval\xcoorend{clip(\xcoorbeg+1*\sc)} 
\draw[very thick](\xcoorbeg,\ycoord) -- (\xcoorend,\ycoord);
\FPeval\result{clip(\xcoorend+0.35*\sc)} 
\ifthenelse{\isempty{#1}}  
{\FPeval\xcoor{\xcoorend}}
{\node at (\result,\ycoord){#1};
\FPeval\xcoor{clip(\xcoorend+0.25*\sc)}}
\node[inner sep=0pt,minimum size=0pt] (#3) at (\xcoorbeg,\ycoord) {};
\node[inner sep=0pt,minimum size=0pt] (#4) at (\xcoorend,\ycoord) {};  
\FPeval\xx{clip(\xcoorbeg-\xvac)} 
\node[inner sep=0pt,minimum size=0pt] (#3v1) at (\xx,\yvac) {}; 
\FPeval\xx{clip(\xcoorbeg+\xvac)} 
\node[inner sep=0pt,minimum size=0pt] (#3v2) at (\xx,\yvac) {};
\FPeval\xx{clip(\xcoorend-\xvac)} 
\node[inner sep=0pt,minimum size=0pt] (#4v1) at (\xx,\yvac) {};
\FPeval\xx{clip(\xcoorend+\xvac)} 
\node[inner sep=0pt,minimum size=0pt] (#4v2) at (\xx,\yvac) {};
\ifthenelse{\isempty{#2}}  {}  
{ 
\FPeval\result{clip((\xcoorend+\xcoorbeg)/2)} 
\setcounter{stoer}{#2} 
\node[inner sep=0pt,minimum size=0pt] at (\result,\ycoord){{\footnotesize\slshape\sffamily \Roman{stoer}}}; }
}

\newcommand{\dAmpThr}[5][]{\FPeval\xcoorbeg{clip(\xcoor+0.5*\sc)} 
\FPeval\xcoorend{clip(\xcoorbeg+1*\sc)} 
\draw[very thick](\xcoorbeg,\ycoord) -- (\xcoorend,\ycoord);
\FPeval\result{clip(\xcoorend+0.35*\sc)} 
\ifthenelse{\isempty{#1}}  
{\FPeval\xcoor{\xcoorend}}
{\node at (\result,\ycoord){#1};
\FPeval\xcoor{clip(\xcoorend+0.25*\sc)}}
\node[inner sep=0pt,minimum size=0pt] (#3) at (\xcoorbeg,\ycoord) {};
\FPeval\result{clip((\xcoorbeg+\xcoorend)/2)} 
\node[inner sep=0pt,minimum size=0pt] (#4) at  (\result,\ycoord) {};
\node[inner sep=0pt,minimum size=0pt] (#5) at (\xcoorend,\ycoord) {};  
\FPeval\xx{clip(\xcoorbeg-\xvac)} 
\node[inner sep=0pt,minimum size=0pt] (#3v1) at (\xx,\yvac) {}; 
\FPeval\xx{clip(\xcoorbeg+\xvac)} 
\node[inner sep=0pt,minimum size=0pt] (#3v2) at (\xx,\yvac) {};
\FPeval\xx{clip(\result-\xvac)} 
\node[inner sep=0pt,minimum size=0pt] (#4v1) at (\xx,\yvac) {};
\FPeval\xx{clip(\result+\xvac)} 
\node[inner sep=0pt,minimum size=0pt] (#4v2) at (\xx,\yvac) {};
\FPeval\xx{clip(\xcoorend-\xvac)} 
\node[inner sep=0pt,minimum size=0pt] (#5v1) at (\xx,\yvac) {};
\FPeval\xx{clip(\xcoorend+\xvac)} 
\node[inner sep=0pt,minimum size=0pt] (#5v2) at (\xx,\yvac) {};
\ifthenelse{\isempty{#2}}  {}  
{ 
\FPeval\result{clip((\xcoorend+\xcoorbeg)/2)} 
\setcounter{stoer}{#2} 
\node[inner sep=0pt,minimum size=0pt] at (\result,\ycoord){{\footnotesize\slshape\sffamily \Roman{stoer}}}; }
}



\newcommand{\dAmpdOne}[3][]{\FPeval\xcoorbeg{clip(\xcoordg+0.5*\sc)} 
\FPeval\xcoorend{clip(\xcoorbeg+0.5*\sc)} 
\draw[very thick](\xcoorbeg,\yvac) -- (\xcoorend,\yvac);
\FPeval\result{clip(\xcoorend+0.35*\sc)} 
\ifthenelse{\isempty{#1}}  
{\FPeval\xcoordg{\xcoorend}}
{\node at (\result,\yvac){#1};
\FPeval\xcoordg{clip(\xcoorend+0.25*\sc)}}
\FPeval\result{clip((\xcoorbeg+\xcoorend)/2)} 
\node[inner sep=0pt,minimum size=0pt] (#3) at  (\result,\yvac) {};
\FPeval\xx{clip(\result-\xvac)} 
\node[inner sep=0pt,minimum size=0pt] (#3v1) at (\xx,\ycoord) {};
\FPeval\xx{clip(\result+\xvac)} 
\node[inner sep=0pt,minimum size=0pt] (#3v2) at (\xx,\ycoord) {};
\ifthenelse{\isempty{#2}}  {}  
{ 
\FPeval\result{clip((\xcoorend+\xcoorbeg)/2)} 
\setcounter{stoer}{#2} 
\node[inner sep=0pt,minimum size=0pt] at (\result,\yvac){{\footnotesize\slshape\sffamily \Roman{stoer}}}; }
}

\newcommand{\dAmpdTwo}[4][]{\FPeval\xcoorbeg{clip(\xcoordg+0.5*\sc)} 
\FPeval\xcoorend{clip(\xcoorbeg+1*\sc)} 
\draw[very thick](\xcoorbeg,\yvac) -- (\xcoorend,\yvac);
\FPeval\result{clip(\xcoorend+0.35*\sc)} 
\ifthenelse{\isempty{#1}}  
{\FPeval\xcoordg{\xcoorend}}
{\node at (\result,\yvac){#1};
\FPeval\xcoordg{clip(\xcoorend+0.25*\sc)}}
\node[inner sep=0pt,minimum size=0pt] (#3) at (\xcoorbeg,\yvac) {};
\node[inner sep=0pt,minimum size=0pt] (#4) at (\xcoorend,\yvac) {};  
\FPeval\xx{clip(\xcoorbeg-\xvac)} 
\node[inner sep=0pt,minimum size=0pt] (#3v1) at (\xx,\ycoord) {}; 
\FPeval\xx{clip(\xcoorbeg+\xvac)} 
\node[inner sep=0pt,minimum size=0pt] (#3v2) at (\xx,\ycoord) {};
\FPeval\xx{clip(\xcoorend-\xvac)} 
\node[inner sep=0pt,minimum size=0pt] (#4v1) at (\xx,\ycoord) {};
\FPeval\xx{clip(\xcoorend+\xvac)} 
\node[inner sep=0pt,minimum size=0pt] (#4v2) at (\xx,\ycoord) {};
\ifthenelse{\isempty{#2}}  {}  
{ 
\FPeval\result{clip((\xcoorend+\xcoorbeg)/2)} 
\setcounter{stoer}{#2} 
\node[inner sep=0pt,minimum size=0pt] at (\result,\yvac){{\footnotesize\slshape\sffamily \Roman{stoer}}}; }
}

\newcommand{\dAmpdThr}[5][]{\FPeval\xcoorbeg{clip(\xcoordg+0.5*\sc)} 
\FPeval\xcoorend{clip(\xcoorbeg+1*\sc)} 
\draw[very thick](\xcoorbeg,\yvac) -- (\xcoorend,\yvac);
\FPeval\result{clip(\xcoorend+0.35*\sc)} 
\ifthenelse{\isempty{#1}}  
{\FPeval\xcoordg{\xcoorend}}
{\node at (\result,\yvac){#1};
\FPeval\xcoordg{clip(\xcoorend+0.25*\sc)}}
\node[inner sep=0pt,minimum size=0pt] (#3) at (\xcoorbeg,\yvac) {};
\FPeval\result{clip((\xcoorbeg+\xcoorend)/2)} 
\node[inner sep=0pt,minimum size=0pt] (#4) at  (\result,\yvac) {};
\node[inner sep=0pt,minimum size=0pt] (#5) at (\xcoorend,\yvac) {};  
\FPeval\xx{clip(\xcoorbeg-\xvac)} 
\node[inner sep=0pt,minimum size=0pt] (#3v1) at (\xx,\ycoord) {}; 
\FPeval\xx{clip(\xcoorbeg+\xvac)} 
\node[inner sep=0pt,minimum size=0pt] (#3v2) at (\xx,\ycoord) {};
\FPeval\xx{clip(\result-\xvac)} 
\node[inner sep=0pt,minimum size=0pt] (#4v1) at (\xx,\ycoord) {};
\FPeval\xx{clip(\result+\xvac)} 
\node[inner sep=0pt,minimum size=0pt] (#4v2) at (\xx,\ycoord) {};
\FPeval\xx{clip(\xcoorend-\xvac)} 
\node[inner sep=0pt,minimum size=0pt] (#5v1) at (\xx,\ycoord) {};
\FPeval\xx{clip(\xcoorend+\xvac)} 
\node[inner sep=0pt,minimum size=0pt] (#5v2) at (\xx,\ycoord) {};
\ifthenelse{\isempty{#2}}  {}  
{ 
\FPeval\result{clip((\xcoorend+\xcoorbeg)/2)} 
\setcounter{stoer}{#2} 
\node[inner sep=0pt,minimum size=0pt] at (\result,\yvac){{\footnotesize\slshape\sffamily \Roman{stoer}}}; }
}

\newcommand{\dAmp}[4][]{%for all
\FPeval\xcoorbeg{clip(\xcoor+0.5*\sc)} 
\FPeval\xcoorend{clip(\xcoorbeg+#3*\sc/2)} 
\draw[exoper_line](\xcoorbeg,\ycoord) -- (\xcoorend,\ycoord);
\FPeval\result{clip(\xcoorend+0.35*\sc)} 
% write label if given
\ifthenelse{\isempty{#1}}  
{\FPeval\xcoor{\xcoorend}}
{\node at (\result,\ycoord){#1};
\FPeval\xcoor{clip(\xcoorend+0.25*\sc)}}
% calculate the vertex-distance (singles are a special case)
\ifthenelse{ #3 = 1}
{ \FPeval\xxx{clip((\xcoorend-\xcoorbeg)/2)} 
  \FPset\result{\xcoorbeg}}
{ \FPeval\xxx{clip((\xcoorend-\xcoorbeg)/( #3 - 1 ))} 
  \FPeval\result{clip(\xcoorbeg-\xxx)}}
% set all nodes (and give them names)
\foreach \x in {1,...,#3}
{
  \coordinate (vertex) at ($(\result,\ycoord)+(\x*\xxx,0)$);
  \node[inner sep=0pt,minimum size=0pt] (#4\x) at  (vertex) {};
  \coordinate (vtxvac) at ($(\result,\yvac)+(\x*\xxx,0)$);
  \node[inner sep=0pt,minimum size=0pt] (#4\x v1) at ($(vtxvac)-(\xvac,0)$) {};
  \node[inner sep=0pt,minimum size=0pt] (#4\x v2) at ($(vtxvac)+(\xvac,0)$) {};
}
\node[inner sep=0pt,minimum size=0pt] (#4) at  (#41) {};
\node[inner sep=0pt,minimum size=0pt] (#4v1) at (#41v1) {};
\node[inner sep=0pt,minimum size=0pt] (#4v2) at (#41v2) {};
% set perturbation (if given)
\ifthenelse{\isempty{#2}}  {}  
{ 
\FPeval\result{clip((\xcoorend+\xcoorbeg)/2)} 
\setcounter{stoer}{#2} 
\node[inner sep=0pt,minimum size=0pt] at (\result,\ycoord){{\footnotesize\slshape\sffamily \Roman{stoer}}}; }
}

\newcommand{\dAmpD}[4][]{%for all
\FPeval\xcoorbeg{clip(\xcoordg+0.5*\sc)} 
\FPeval\xcoorend{clip(\xcoorbeg+#3*\sc/2)} 
\draw[exoper_line](\xcoorbeg,\yvac) -- (\xcoorend,\yvac);
\FPeval\result{clip(\xcoorend+0.35*\sc)} 
% write label if given
\ifthenelse{\isempty{#1}}  
{\FPeval\xcoordg{\xcoorend}}
{\node at (\result,\yvac){#1};
\FPeval\xcoordg{clip(\xcoorend+0.25*\sc)}}
% calculate the vertex-distance (singles are a special case)
\ifthenelse{ #3 = 1}
{ \FPeval\xxx{clip((\xcoorend-\xcoorbeg)/2)} 
  \FPset\result{\xcoorbeg}}
{ \FPeval\xxx{clip((\xcoorend-\xcoorbeg)/( #3 - 1 ))} 
  \FPeval\result{clip(\xcoorbeg-\xxx)}}
% set all nodes (and give them names)
\foreach \x in {1,...,#3}
{
  \coordinate (vertex) at ($(\result,\yvac)+(\x*\xxx,0)$);
  \node[inner sep=0pt,minimum size=0pt] (#4\x) at  (vertex) {};
  \coordinate (vtxvac) at ($(\result,\ycoord)+(\x*\xxx,0)$);
  \node[inner sep=0pt,minimum size=0pt] (#4\x v1) at ($(vtxvac)-(\xvac,0)$) {};
  \node[inner sep=0pt,minimum size=0pt] (#4\x v2) at ($(vtxvac)+(\xvac,0)$) {};
}
\node[inner sep=0pt,minimum size=0pt] (#4) at  (#41) {};
\node[inner sep=0pt,minimum size=0pt] (#4v1) at (#41v1) {};
\node[inner sep=0pt,minimum size=0pt] (#4v2) at (#41v2) {};
% set perturbation (if given)
\ifthenelse{\isempty{#2}}  {}  
{ 
\FPeval\result{clip((\xcoorend+\xcoorbeg)/2)} 
\setcounter{stoer}{#2} 
\node[inner sep=0pt,minimum size=0pt] at (\result,\yvac){{\footnotesize\slshape\sffamily \Roman{stoer}}}; }
}

\newcommand{\dHone}[2][]{\FPeval\xcoorbeg{clip(\xcoorH+0.5*\sc)}
\FPeval\xcoorend{clip(\xcoorbeg+0.8*\sc)} 
\draw[hoper_line](\xcoorbeg,\ycoor) -- (\xcoorend,\ycoor);
\node at (\xcoorend,\ycoor){$\bf \times$};
\FPeval\result{clip(\xcoorend+0.35*\sc)} 
%name of operator
\ifthenelse{\isempty{#1}}
{\FPeval\xcoorH{\xcoorend}}
{\node at (\result,\ycoor){#1};
\FPeval\xcoorH{clip(\xcoorend+0.25*\sc)}}
\node[inner sep=0pt,minimum size=0pt] (#2) at (\xcoorbeg,\ycoor){}; 
\FPeval\xx{clip(\xcoorbeg-\xvac/2)} 
\node[inner sep=0pt,minimum size=0pt] (#2v1) at (\xx,\yvac) {}; 
\FPeval\xx{clip(\xcoorbeg+\xvac/2)} 
\node[inner sep=0pt,minimum size=0pt] (#2v2) at (\xx,\yvac) {};
}

\newcommand{\dHtwo}[3][]{ \FPeval\xcoorbeg{clip(\xcoorH+0.5*\sc)}
\FPeval\xcoorend{clip(\xcoorbeg+1*\sc)} 
\draw[hoper_line](\xcoorbeg,\ycoor) -- (\xcoorend,\ycoor); 
\FPeval\result{clip(\xcoorend+0.35*\sc)} 
%name of operator
\ifthenelse{\isempty{#1}}
{\FPeval\xcoorH{\xcoorend}}
{\node at (\result,\ycoor){#1};
\FPeval\xcoorH{clip(\xcoorend+0.25*\sc)}}
\node[inner sep=0pt,minimum size=0pt] (#2) at (\xcoorbeg,\ycoor) {};
\node[inner sep=0pt,minimum size=0pt] (#3) at (\xcoorend,\ycoor) {};
\FPeval\xx{clip(\xcoorbeg-\xvac/2)} 
\node[inner sep=0pt,minimum size=0pt] (#2v1) at (\xx,\yvac) {};
\FPeval\xx{clip(\xcoorbeg+\xvac/2)} 
\node[inner sep=0pt,minimum size=0pt] (#2v2) at (\xx,\yvac) {}; 
\FPeval\xx{clip(\xcoorend-\xvac/2)} 
\node[inner sep=0pt,minimum size=0pt] (#3v1) at (\xx,\yvac) {};
\FPeval\xx{clip(\xcoorend+\xvac/2)} 
\node[inner sep=0pt,minimum size=0pt] (#3v2) at (\xx,\yvac) {};
} 

\newcommand{\dTone}[2][]{\dAmpOne[$_{T_1}$]{#1}{#2}}
\newcommand{\dTones}[2][]{\dAmpOne{#1}{#2}}
\newcommand{\dUone}[2][]{\dAmpOne[$_{U_1}$]{#1}{#2}}

\newcommand{\dTtwo}[3][]{\dAmpTwo[$_{T_2}$]{#1}{#2}{#3}}
\newcommand{\dTtwos}[3][]{\dAmpTwo{#1}{#2}{#3}}
\newcommand{\dUtwo}[3][]{\dAmpTwo[$_{U_2}$]{#1}{#2}{#3}}

\newcommand{\dTthr}[4][]{\dAmpThr[$_{T_3}$]{#1}{#2}{#3}{#4}}
\newcommand{\dTthrs}[4][]{\dAmpThr{#1}{#2}{#3}{#4}}
\newcommand{\dUthr}[4][]{\dAmpThr[$_{U_3}$]{#1}{#2}{#3}{#4}}

\newcommand{\dTdone}[2][]{\dAmpdOne[$_{T^{\dagger}_1}$]{#1}{#2}}
\newcommand{\dTdones}[2][]{\dAmpdOne{#1}{#2}}
\newcommand{\dUdone}[2][]{\dAmpdOne[$_{U^{\dagger}_1}$]{#1}{#2}}

\newcommand{\dTdtwo}[3][]{\dAmpdTwo[$_{T^{\dagger}_2}$]{#1}{#2}{#3}}
\newcommand{\dTdtwos}[3][]{\dAmpdTwo{#1}{#2}{#3}}
\newcommand{\dUdtwo}[3][]{\dAmpdTwo[$_{U^{\dagger}_2}$]{#1}{#2}{#3}}

\newcommand{\dTdthr}[4][]{\dAmpdThr[$_{T^{\dagger}_3}$]{#1}{#2}{#3}{#4}}
\newcommand{\dTdthrs}[4][]{\dAmpdThr{#1}{#2}{#3}{#4}}
\newcommand{\dUdthr}[4][]{\dAmpdThr[$_{U^{\dagger}_3}$]{#1}{#2}{#3}{#4}}

%general excitations
\newcommand{\dT}[3][]{\dAmp[$_{T_#2}$]{#1}{#2}{#3}}
\newcommand{\dTs}[3][]{\dAmp{#1}{#2}{#3}}
\newcommand{\dU}[3][]{\dAmp[$_{U_#2}$]{#1}{#2}{#3}}
\newcommand{\dTv}[3][]{
\tikzset{exoper_line/.style={exvac_line_save}}
\dAmp{#1}{#2}{#3}
\tikzset{exoper_line/.style={exoper_line_save}} }

\newcommand{\dTd}[3][]{\dAmpD[$_{T^{\dagger}_#2}$]{#1}{#2}{#3}}
\newcommand{\dTds}[3][]{\dAmpD{#1}{#2}{#3}}
\newcommand{\dUd}[3][]{\dAmpD[$_{U^{\dagger}_#2}$]{#1}{#2}{#3}}
\newcommand{\dTdv}[3][]{
\tikzset{exoper_line/.style={exvac_line_save}}
\dAmpD{#1}{#2}{#3}
\tikzset{exoper_line/.style={exoper_line_save}} }

%Hamilton parts
\newcommand{\dF}[1]{\dHone[$_{F}$]{#1}}
\newcommand{\dFs}[1]{\dHone{#1}}
\newcommand{\dX}[1]{\dHone[$_{X}$]{#1}}

\newcommand{\dW}[2]{\dHtwo[$_{W}$]{#1}{#2}}
\newcommand{\dWs}[2]{\dHtwo{#1}{#2}}
\newcommand{\dXtwo}[2]{\dHtwo[$_{X}$]{#1}{#2}}

\newcommand{\dline}[3][]{
\ifthenelse{\isempty{#1}}
{\draw[ph_line](#2) -- (#3);}
{\draw[ph_line](#2) -- node[inner sep=0pt,minimum size=0pt,right = 0.5pt] {\footnotesize\slshape\sffamily #1} (#3);}
}
\newcommand{\dcurve}[3][]{
\ifthenelse{\isempty{#1}}
{\draw[ph_line](#2) to [bend right=30] (#3);}
{\draw[ph_line](#2) to [bend right=30] node[inner sep=0pt,minimum size=0pt,right = 0.5pt] {\footnotesize\slshape\sffamily #1} (#3);}
}
\newcommand{\dcurver}[3][]{
\ifthenelse{\isempty{#1}}
{\draw[ph_line](#2) to [bend left=30] (#3);}
{\draw[ph_line](#2) to [bend left=30] node[inner sep=0pt,minimum size=0pt,right = 0.5pt] {\footnotesize\slshape\sffamily #1} (#3);}
}
\newcommand{\dcurcur}[2]{\dcurve{#1}{#2} \dcurve{#2}{#1}}
\newcommand{\dmovex}[1]{\FPeval\xcoorH{clip(\xcoord+#1*0.25*\sc)}}
\newcommand{\dmoveT}[1]{\FPeval\xcoor{clip(\xcoord+#1*0.25*\sc)}}
\newcommand{\dmoveTd}[1]{\FPeval\xcoordg{clip(\xcoord+#1*0.25*\sc)}}
\newcommand{\dmovac}[1]{\FPeval\yvac{clip(\yvac+#1*0.25*\sc)}}
\newcommand{\dname}[1]{
\FPeval\xx{clip(\xcoord+1*\sc)} \FPeval\result{clip(\yvac+0.25*\sc)} 
\node at (\xx,\result) {#1};}
\newcommand{\dtext}[2]{\FPeval\xx{clip(\xcoord+#1*\sc)} \FPeval\result{clip((\ycoord+\yvac)/2)} 
\node at (\xx,\result) {#2};}

\newcommand{\dscale}[1]{\FPset\sc{#1} 
\FPeval\xcoord{clip(-1*\sc)} \FPeval\ycoord{clip(-1*\sc)} 
\FPset\xcoor{\xcoord} \FPeval\ycoor{clip(\ycoord+1*\sc)} 
\FPset\xcoordg{\xcoord}
\FPset\xcoorH{\xcoord}
\FPeval\yvac{clip(\yvac*\sc)} \FPeval\xvac{clip(\xvac*\sc)}}
% end of Diagram-package


% test:
% \begin{pspicture}(\xcoord,\ycoord)(3,3)
% \dTone{t1}
% \dTtwo{t21}{t22}
% \FPeval\xcoorbeg{clip(\xcoord+1)}
% \dV{vn1}{vn2}
% \dline{t1}{vn1}
% \dline{t1v1}{t1}
% \dline{t21}{vn2}
% \dline{t21}{t2v1}
% \dline{vn1}{vn1v2}
% \dline{vn2}{vn2v2}
% \dline{t22v1}{t22}
% \dline{t22}{t22v2}
% \end{pspicture}

%\pgfrealjobname{manual}
\dsavediags{manual}
%

\usepackage{amsmath,amsfonts,amssymb}
\usepackage{listings}
\usepackage{hyperref}
\hypersetup{
    bookmarks=true,         % show bookmarks bar?
    unicode=false,          % non-Latin characters in Acrobat’s bookmarks
    pdftoolbar=true,        % show Acrobat’s toolbar?
    pdfmenubar=true,        % show Acrobat’s menu?
    colorlinks=true,        % false: boxed links; true: colored links
    linkcolor=red,          % color of internal links
    citecolor=green,        % color of links to bibliography
    filecolor=magenta,      % color of file links
    urlcolor=cyan           % color of external links
}

\include{definitions}
\newcommand{\myind}{\hspace{10pt}}

\begin{document}
\author{Daniel Kats}
\title{CCDiag \\ \normalsize git: 8534b78a5e0d942}

\maketitle

\tableofcontents

\section{Using CCDiag}

CCDiag is a TeX file, which allows you to simply draw Coupled Cluster diagrams using TikZ/PGF.
Only the {\it TikZ} package is required; i.e. to include CCDiag in your TeX file use
\lstset{language=[LaTeX]Tex, frame=shadowbox, rulesepcolor=\color{gray}}
\begin{lstlisting}
%ccdiag
\usepackage{tikz}
%
% \usepackage{tikz}
% \usepackage{xifthen}
% \usepackage{fp}
\usetikzlibrary{calc}
%
%       new commands (-> Diagram-package)
%           D.Kats, September 2011
%
% \bdiag[s] start diagram. If s is given, the H diagram is in the middle (otherwise it is shifted to the vacuum)
%
% \dT[<order>]{<exc.level>}{<node>} -> T_<exc.level>^{(<order>)}
% node-names are generated as <node>1, <node>2, <node>3, ...
% can be used without <order> (\dT{<exc.level>}{<node>})
% same with \dU (will print U_<exc.level> as label)
% for excitation operators without label use \dTs
% One can customize labels (see e.g. how \dU is defined)
% \dTd, \dTds, \dUd are T^{\dagger}, {}^{\dagger}, U^{\dagger}
%
% \dTdv[<order>]{<exc.level>}{<node>} : draw vacuum explicitly (\tau^\dagger) 
% \dTv[<order>]{<exc.level>}{<node>} -> \tau
%
% \dF{<node>} -> F
% \dFs{<node>} : one-electron operator without label
% \dX{<node>} : one-electron perturbation (X as label)
% \dHone[<name>]{<node>} : one-electron operator with label <name> 
% \dW{<left node>}{<right node>} -> W
% \dWs{<left node>}{<right node>} : two-electron operator without label
% \dXtwo{<left node>}{<right node>} : two-electron perturbation (X as label)
% \dHtwo[<name>]{<left node>}{<right node>} : two-electron operator with label <name>
%
% \dline[<index>]{<from node>}{<to node>} ->    "--->" (if <index> given - write <index> to the right of the line)
% \dcurve[<index>]{<from node>}{<to node>} ->   curved "--->" 
% \dcurver[<index>]{<from node>}{<to node>} ->   curved "--->" (reverse bend)
% \dcurcur{<node1>}{<node2>} ->     cycled curved "--->"
%  left vacuum-node for <node1> is called <node1>v1
%  right -------------"----------------   <node1>v2
%
% \dmovex{<value>} -> move W or F horizontally
% \dmoveT{<value>} -> move T horizontally
% \dmovac{<value>} -> move vacuum (and daggers) vertically 
% \dmoveTd{<value>} -> move T^\dagger horizontally
%
% \dscale{<value>} -> scale size of diagrams with <value>
%
% \dname{<text>} -> write <text> over the diagram
% \dtext{<text>} -> write <text> in the diagram
%
% change exoper_line_save, exvac_line_save, hoper_line_save, ph_line_save
%  in order to change excitation operator, explicit vacuum, H-operator, or p/h line styles.
%
% Old commands (useful for custom node-names):
% \dTone{<node>} -> T_1
% \dTtwo{<left node>}{<right node>} -> T_2
% \dTthr{<left node>}{<middle node>}{<right node>} -> T_3
%
% \dTone[<order>]{<node>} -> T_1^{(<order>)}
% \dTtwo[<order>]{<left node>}{<right node>} -> T_2^{(<order>)}
% \dTthr[<order>]{<left node>}{<middle node>}{<right node>} -> T_3^{(<order>)}
%
% same with \dUone, \dUtwo, \dUthr (will print U_1, U_2 or U_3)
% for excitation operators without label use \dTones, \dTtwos, \dTthrs
% One can customize labels (see e.g. how \dUone is defined)


\edef\xcoord{11} \edef\xcoor{11} \edef\xcoorbeg{11} \edef\xcoorend{11} \edef\result{11} \edef\ycoor{11} \edef\ycoord{11} \edef\sc{11} 
\FPset\sc{1} %default scale-value
\FPeval\xcoord{clip(-1*\sc)} \FPeval\ycoord{clip(-1*\sc)} 
\FPset\xcoor{\xcoord} \FPeval\ycoor{clip(\ycoord+1*\sc)} 
\FPset\xcoordg{\xcoord}
\FPset\xcoorH{\xcoord}
\FPeval\yvac{clip(\ycoor+0.5*\sc)} \FPeval\xvac{clip(0.25*\sc)}
\newcounter{stoer} 
\tikzset{exoper_line_save/.style={very thick, solid}}
\tikzset{exvac_line_save/.style={very thick, dotted}}
\tikzset{hoper_line_save/.style={very thick, dashed}}
\tikzset{ph_line_save/.style={->,>=stealth,thin}}
\tikzset{exoper_line/.style={exoper_line_save}}
\tikzset{hoper_line/.style={hoper_line_save}}
\tikzset{ph_line/.style={ph_line_save}}

\newcommand{\bdiag}[1][]{
  \begin{tikzpicture}
    \ifthenelse{\isempty{#1}}
      {}
      {%symmetric diagram
        \FPeval\yvac{clip(\ycoor+1*\sc)}
        \FPeval\xvac{clip(0.5*\sc)}
      }
}
\newcommand{\ediag}{\end{tikzpicture}}

\newcommand{\dAmpOne}[3][]{\FPeval\xcoorbeg{clip(\xcoor+0.5*\sc)} 
\FPeval\xcoorend{clip(\xcoorbeg+0.5*\sc)} 
\draw[very thick](\xcoorbeg,\ycoord) -- (\xcoorend,\ycoord);
\FPeval\result{clip(\xcoorend+0.35*\sc)} 
\ifthenelse{\isempty{#1}}  
{\FPeval\xcoor{\xcoorend}}
{\node at (\result,\ycoord){#1};
\FPeval\xcoor{clip(\xcoorend+0.25*\sc)}} 
\FPeval\result{clip((\xcoorbeg+\xcoorend)/2)} 
\node[inner sep=0pt,minimum size=0pt] (#3) at  (\result,\ycoord) {};
\FPeval\xx{clip(\xcoorbeg-\xvac)} 
\node[inner sep=0pt,minimum size=0pt] (#3v1) at (\xx,\yvac) {}; 
\FPeval\xx{clip(\xcoorbeg+\xvac)} 
\node[inner sep=0pt,minimum size=0pt] (#3v2) at (\xx,\yvac) {};
\ifthenelse{\isempty{#2}}  {}  
{ 
\FPeval\result{clip((\xcoorend+\xcoorbeg)/2)} 
\setcounter{stoer}{#2} 
\node[inner sep=0pt,minimum size=0pt] at (\result,\ycoord){{\footnotesize\slshape\sffamily \Roman{stoer}}}; }
}

\newcommand{\dAmpTwo}[4][]{\FPeval\xcoorbeg{clip(\xcoor+0.5*\sc)} 
\FPeval\xcoorend{clip(\xcoorbeg+1*\sc)} 
\draw[very thick](\xcoorbeg,\ycoord) -- (\xcoorend,\ycoord);
\FPeval\result{clip(\xcoorend+0.35*\sc)} 
\ifthenelse{\isempty{#1}}  
{\FPeval\xcoor{\xcoorend}}
{\node at (\result,\ycoord){#1};
\FPeval\xcoor{clip(\xcoorend+0.25*\sc)}}
\node[inner sep=0pt,minimum size=0pt] (#3) at (\xcoorbeg,\ycoord) {};
\node[inner sep=0pt,minimum size=0pt] (#4) at (\xcoorend,\ycoord) {};  
\FPeval\xx{clip(\xcoorbeg-\xvac)} 
\node[inner sep=0pt,minimum size=0pt] (#3v1) at (\xx,\yvac) {}; 
\FPeval\xx{clip(\xcoorbeg+\xvac)} 
\node[inner sep=0pt,minimum size=0pt] (#3v2) at (\xx,\yvac) {};
\FPeval\xx{clip(\xcoorend-\xvac)} 
\node[inner sep=0pt,minimum size=0pt] (#4v1) at (\xx,\yvac) {};
\FPeval\xx{clip(\xcoorend+\xvac)} 
\node[inner sep=0pt,minimum size=0pt] (#4v2) at (\xx,\yvac) {};
\ifthenelse{\isempty{#2}}  {}  
{ 
\FPeval\result{clip((\xcoorend+\xcoorbeg)/2)} 
\setcounter{stoer}{#2} 
\node[inner sep=0pt,minimum size=0pt] at (\result,\ycoord){{\footnotesize\slshape\sffamily \Roman{stoer}}}; }
}

\newcommand{\dAmpThr}[5][]{\FPeval\xcoorbeg{clip(\xcoor+0.5*\sc)} 
\FPeval\xcoorend{clip(\xcoorbeg+1*\sc)} 
\draw[very thick](\xcoorbeg,\ycoord) -- (\xcoorend,\ycoord);
\FPeval\result{clip(\xcoorend+0.35*\sc)} 
\ifthenelse{\isempty{#1}}  
{\FPeval\xcoor{\xcoorend}}
{\node at (\result,\ycoord){#1};
\FPeval\xcoor{clip(\xcoorend+0.25*\sc)}}
\node[inner sep=0pt,minimum size=0pt] (#3) at (\xcoorbeg,\ycoord) {};
\FPeval\result{clip((\xcoorbeg+\xcoorend)/2)} 
\node[inner sep=0pt,minimum size=0pt] (#4) at  (\result,\ycoord) {};
\node[inner sep=0pt,minimum size=0pt] (#5) at (\xcoorend,\ycoord) {};  
\FPeval\xx{clip(\xcoorbeg-\xvac)} 
\node[inner sep=0pt,minimum size=0pt] (#3v1) at (\xx,\yvac) {}; 
\FPeval\xx{clip(\xcoorbeg+\xvac)} 
\node[inner sep=0pt,minimum size=0pt] (#3v2) at (\xx,\yvac) {};
\FPeval\xx{clip(\result-\xvac)} 
\node[inner sep=0pt,minimum size=0pt] (#4v1) at (\xx,\yvac) {};
\FPeval\xx{clip(\result+\xvac)} 
\node[inner sep=0pt,minimum size=0pt] (#4v2) at (\xx,\yvac) {};
\FPeval\xx{clip(\xcoorend-\xvac)} 
\node[inner sep=0pt,minimum size=0pt] (#5v1) at (\xx,\yvac) {};
\FPeval\xx{clip(\xcoorend+\xvac)} 
\node[inner sep=0pt,minimum size=0pt] (#5v2) at (\xx,\yvac) {};
\ifthenelse{\isempty{#2}}  {}  
{ 
\FPeval\result{clip((\xcoorend+\xcoorbeg)/2)} 
\setcounter{stoer}{#2} 
\node[inner sep=0pt,minimum size=0pt] at (\result,\ycoord){{\footnotesize\slshape\sffamily \Roman{stoer}}}; }
}



\newcommand{\dAmpdOne}[3][]{\FPeval\xcoorbeg{clip(\xcoordg+0.5*\sc)} 
\FPeval\xcoorend{clip(\xcoorbeg+0.5*\sc)} 
\draw[very thick](\xcoorbeg,\yvac) -- (\xcoorend,\yvac);
\FPeval\result{clip(\xcoorend+0.35*\sc)} 
\ifthenelse{\isempty{#1}}  
{\FPeval\xcoordg{\xcoorend}}
{\node at (\result,\yvac){#1};
\FPeval\xcoordg{clip(\xcoorend+0.25*\sc)}}
\FPeval\result{clip((\xcoorbeg+\xcoorend)/2)} 
\node[inner sep=0pt,minimum size=0pt] (#3) at  (\result,\yvac) {};
\FPeval\xx{clip(\result-\xvac)} 
\node[inner sep=0pt,minimum size=0pt] (#3v1) at (\xx,\ycoord) {};
\FPeval\xx{clip(\result+\xvac)} 
\node[inner sep=0pt,minimum size=0pt] (#3v2) at (\xx,\ycoord) {};
\ifthenelse{\isempty{#2}}  {}  
{ 
\FPeval\result{clip((\xcoorend+\xcoorbeg)/2)} 
\setcounter{stoer}{#2} 
\node[inner sep=0pt,minimum size=0pt] at (\result,\yvac){{\footnotesize\slshape\sffamily \Roman{stoer}}}; }
}

\newcommand{\dAmpdTwo}[4][]{\FPeval\xcoorbeg{clip(\xcoordg+0.5*\sc)} 
\FPeval\xcoorend{clip(\xcoorbeg+1*\sc)} 
\draw[very thick](\xcoorbeg,\yvac) -- (\xcoorend,\yvac);
\FPeval\result{clip(\xcoorend+0.35*\sc)} 
\ifthenelse{\isempty{#1}}  
{\FPeval\xcoordg{\xcoorend}}
{\node at (\result,\yvac){#1};
\FPeval\xcoordg{clip(\xcoorend+0.25*\sc)}}
\node[inner sep=0pt,minimum size=0pt] (#3) at (\xcoorbeg,\yvac) {};
\node[inner sep=0pt,minimum size=0pt] (#4) at (\xcoorend,\yvac) {};  
\FPeval\xx{clip(\xcoorbeg-\xvac)} 
\node[inner sep=0pt,minimum size=0pt] (#3v1) at (\xx,\ycoord) {}; 
\FPeval\xx{clip(\xcoorbeg+\xvac)} 
\node[inner sep=0pt,minimum size=0pt] (#3v2) at (\xx,\ycoord) {};
\FPeval\xx{clip(\xcoorend-\xvac)} 
\node[inner sep=0pt,minimum size=0pt] (#4v1) at (\xx,\ycoord) {};
\FPeval\xx{clip(\xcoorend+\xvac)} 
\node[inner sep=0pt,minimum size=0pt] (#4v2) at (\xx,\ycoord) {};
\ifthenelse{\isempty{#2}}  {}  
{ 
\FPeval\result{clip((\xcoorend+\xcoorbeg)/2)} 
\setcounter{stoer}{#2} 
\node[inner sep=0pt,minimum size=0pt] at (\result,\yvac){{\footnotesize\slshape\sffamily \Roman{stoer}}}; }
}

\newcommand{\dAmpdThr}[5][]{\FPeval\xcoorbeg{clip(\xcoordg+0.5*\sc)} 
\FPeval\xcoorend{clip(\xcoorbeg+1*\sc)} 
\draw[very thick](\xcoorbeg,\yvac) -- (\xcoorend,\yvac);
\FPeval\result{clip(\xcoorend+0.35*\sc)} 
\ifthenelse{\isempty{#1}}  
{\FPeval\xcoordg{\xcoorend}}
{\node at (\result,\yvac){#1};
\FPeval\xcoordg{clip(\xcoorend+0.25*\sc)}}
\node[inner sep=0pt,minimum size=0pt] (#3) at (\xcoorbeg,\yvac) {};
\FPeval\result{clip((\xcoorbeg+\xcoorend)/2)} 
\node[inner sep=0pt,minimum size=0pt] (#4) at  (\result,\yvac) {};
\node[inner sep=0pt,minimum size=0pt] (#5) at (\xcoorend,\yvac) {};  
\FPeval\xx{clip(\xcoorbeg-\xvac)} 
\node[inner sep=0pt,minimum size=0pt] (#3v1) at (\xx,\ycoord) {}; 
\FPeval\xx{clip(\xcoorbeg+\xvac)} 
\node[inner sep=0pt,minimum size=0pt] (#3v2) at (\xx,\ycoord) {};
\FPeval\xx{clip(\result-\xvac)} 
\node[inner sep=0pt,minimum size=0pt] (#4v1) at (\xx,\ycoord) {};
\FPeval\xx{clip(\result+\xvac)} 
\node[inner sep=0pt,minimum size=0pt] (#4v2) at (\xx,\ycoord) {};
\FPeval\xx{clip(\xcoorend-\xvac)} 
\node[inner sep=0pt,minimum size=0pt] (#5v1) at (\xx,\ycoord) {};
\FPeval\xx{clip(\xcoorend+\xvac)} 
\node[inner sep=0pt,minimum size=0pt] (#5v2) at (\xx,\ycoord) {};
\ifthenelse{\isempty{#2}}  {}  
{ 
\FPeval\result{clip((\xcoorend+\xcoorbeg)/2)} 
\setcounter{stoer}{#2} 
\node[inner sep=0pt,minimum size=0pt] at (\result,\yvac){{\footnotesize\slshape\sffamily \Roman{stoer}}}; }
}

\newcommand{\dAmp}[4][]{%for all
\FPeval\xcoorbeg{clip(\xcoor+0.5*\sc)} 
\FPeval\xcoorend{clip(\xcoorbeg+#3*\sc/2)} 
\draw[exoper_line](\xcoorbeg,\ycoord) -- (\xcoorend,\ycoord);
\FPeval\result{clip(\xcoorend+0.35*\sc)} 
% write label if given
\ifthenelse{\isempty{#1}}  
{\FPeval\xcoor{\xcoorend}}
{\node at (\result,\ycoord){#1};
\FPeval\xcoor{clip(\xcoorend+0.25*\sc)}}
% calculate the vertex-distance (singles are a special case)
\ifthenelse{ #3 = 1}
{ \FPeval\xxx{clip((\xcoorend-\xcoorbeg)/2)} 
  \FPset\result{\xcoorbeg}}
{ \FPeval\xxx{clip((\xcoorend-\xcoorbeg)/( #3 - 1 ))} 
  \FPeval\result{clip(\xcoorbeg-\xxx)}}
% set all nodes (and give them names)
\foreach \x in {1,...,#3}
{
  \coordinate (vertex) at ($(\result,\ycoord)+(\x*\xxx,0)$);
  \node[inner sep=0pt,minimum size=0pt] (#4\x) at  (vertex) {};
  \coordinate (vtxvac) at ($(\result,\yvac)+(\x*\xxx,0)$);
  \node[inner sep=0pt,minimum size=0pt] (#4\x v1) at ($(vtxvac)-(\xvac,0)$) {};
  \node[inner sep=0pt,minimum size=0pt] (#4\x v2) at ($(vtxvac)+(\xvac,0)$) {};
}
\node[inner sep=0pt,minimum size=0pt] (#4) at  (#41) {};
\node[inner sep=0pt,minimum size=0pt] (#4v1) at (#41v1) {};
\node[inner sep=0pt,minimum size=0pt] (#4v2) at (#41v2) {};
% set perturbation (if given)
\ifthenelse{\isempty{#2}}  {}  
{ 
\FPeval\result{clip((\xcoorend+\xcoorbeg)/2)} 
\setcounter{stoer}{#2} 
\node[inner sep=0pt,minimum size=0pt] at (\result,\ycoord){{\footnotesize\slshape\sffamily \Roman{stoer}}}; }
}

\newcommand{\dAmpD}[4][]{%for all
\FPeval\xcoorbeg{clip(\xcoordg+0.5*\sc)} 
\FPeval\xcoorend{clip(\xcoorbeg+#3*\sc/2)} 
\draw[exoper_line](\xcoorbeg,\yvac) -- (\xcoorend,\yvac);
\FPeval\result{clip(\xcoorend+0.35*\sc)} 
% write label if given
\ifthenelse{\isempty{#1}}  
{\FPeval\xcoordg{\xcoorend}}
{\node at (\result,\yvac){#1};
\FPeval\xcoordg{clip(\xcoorend+0.25*\sc)}}
% calculate the vertex-distance (singles are a special case)
\ifthenelse{ #3 = 1}
{ \FPeval\xxx{clip((\xcoorend-\xcoorbeg)/2)} 
  \FPset\result{\xcoorbeg}}
{ \FPeval\xxx{clip((\xcoorend-\xcoorbeg)/( #3 - 1 ))} 
  \FPeval\result{clip(\xcoorbeg-\xxx)}}
% set all nodes (and give them names)
\foreach \x in {1,...,#3}
{
  \coordinate (vertex) at ($(\result,\yvac)+(\x*\xxx,0)$);
  \node[inner sep=0pt,minimum size=0pt] (#4\x) at  (vertex) {};
  \coordinate (vtxvac) at ($(\result,\ycoord)+(\x*\xxx,0)$);
  \node[inner sep=0pt,minimum size=0pt] (#4\x v1) at ($(vtxvac)-(\xvac,0)$) {};
  \node[inner sep=0pt,minimum size=0pt] (#4\x v2) at ($(vtxvac)+(\xvac,0)$) {};
}
\node[inner sep=0pt,minimum size=0pt] (#4) at  (#41) {};
\node[inner sep=0pt,minimum size=0pt] (#4v1) at (#41v1) {};
\node[inner sep=0pt,minimum size=0pt] (#4v2) at (#41v2) {};
% set perturbation (if given)
\ifthenelse{\isempty{#2}}  {}  
{ 
\FPeval\result{clip((\xcoorend+\xcoorbeg)/2)} 
\setcounter{stoer}{#2} 
\node[inner sep=0pt,minimum size=0pt] at (\result,\yvac){{\footnotesize\slshape\sffamily \Roman{stoer}}}; }
}

\newcommand{\dHone}[2][]{\FPeval\xcoorbeg{clip(\xcoorH+0.5*\sc)}
\FPeval\xcoorend{clip(\xcoorbeg+0.8*\sc)} 
\draw[hoper_line](\xcoorbeg,\ycoor) -- (\xcoorend,\ycoor);
\node at (\xcoorend,\ycoor){$\bf \times$};
\FPeval\result{clip(\xcoorend+0.35*\sc)} 
%name of operator
\ifthenelse{\isempty{#1}}
{\FPeval\xcoorH{\xcoorend}}
{\node at (\result,\ycoor){#1};
\FPeval\xcoorH{clip(\xcoorend+0.25*\sc)}}
\node[inner sep=0pt,minimum size=0pt] (#2) at (\xcoorbeg,\ycoor){}; 
\FPeval\xx{clip(\xcoorbeg-\xvac/2)} 
\node[inner sep=0pt,minimum size=0pt] (#2v1) at (\xx,\yvac) {}; 
\FPeval\xx{clip(\xcoorbeg+\xvac/2)} 
\node[inner sep=0pt,minimum size=0pt] (#2v2) at (\xx,\yvac) {};
}

\newcommand{\dHtwo}[3][]{ \FPeval\xcoorbeg{clip(\xcoorH+0.5*\sc)}
\FPeval\xcoorend{clip(\xcoorbeg+1*\sc)} 
\draw[hoper_line](\xcoorbeg,\ycoor) -- (\xcoorend,\ycoor); 
\FPeval\result{clip(\xcoorend+0.35*\sc)} 
%name of operator
\ifthenelse{\isempty{#1}}
{\FPeval\xcoorH{\xcoorend}}
{\node at (\result,\ycoor){#1};
\FPeval\xcoorH{clip(\xcoorend+0.25*\sc)}}
\node[inner sep=0pt,minimum size=0pt] (#2) at (\xcoorbeg,\ycoor) {};
\node[inner sep=0pt,minimum size=0pt] (#3) at (\xcoorend,\ycoor) {};
\FPeval\xx{clip(\xcoorbeg-\xvac/2)} 
\node[inner sep=0pt,minimum size=0pt] (#2v1) at (\xx,\yvac) {};
\FPeval\xx{clip(\xcoorbeg+\xvac/2)} 
\node[inner sep=0pt,minimum size=0pt] (#2v2) at (\xx,\yvac) {}; 
\FPeval\xx{clip(\xcoorend-\xvac/2)} 
\node[inner sep=0pt,minimum size=0pt] (#3v1) at (\xx,\yvac) {};
\FPeval\xx{clip(\xcoorend+\xvac/2)} 
\node[inner sep=0pt,minimum size=0pt] (#3v2) at (\xx,\yvac) {};
} 

\newcommand{\dTone}[2][]{\dAmpOne[$_{T_1}$]{#1}{#2}}
\newcommand{\dTones}[2][]{\dAmpOne{#1}{#2}}
\newcommand{\dUone}[2][]{\dAmpOne[$_{U_1}$]{#1}{#2}}

\newcommand{\dTtwo}[3][]{\dAmpTwo[$_{T_2}$]{#1}{#2}{#3}}
\newcommand{\dTtwos}[3][]{\dAmpTwo{#1}{#2}{#3}}
\newcommand{\dUtwo}[3][]{\dAmpTwo[$_{U_2}$]{#1}{#2}{#3}}

\newcommand{\dTthr}[4][]{\dAmpThr[$_{T_3}$]{#1}{#2}{#3}{#4}}
\newcommand{\dTthrs}[4][]{\dAmpThr{#1}{#2}{#3}{#4}}
\newcommand{\dUthr}[4][]{\dAmpThr[$_{U_3}$]{#1}{#2}{#3}{#4}}

\newcommand{\dTdone}[2][]{\dAmpdOne[$_{T^{\dagger}_1}$]{#1}{#2}}
\newcommand{\dTdones}[2][]{\dAmpdOne{#1}{#2}}
\newcommand{\dUdone}[2][]{\dAmpdOne[$_{U^{\dagger}_1}$]{#1}{#2}}

\newcommand{\dTdtwo}[3][]{\dAmpdTwo[$_{T^{\dagger}_2}$]{#1}{#2}{#3}}
\newcommand{\dTdtwos}[3][]{\dAmpdTwo{#1}{#2}{#3}}
\newcommand{\dUdtwo}[3][]{\dAmpdTwo[$_{U^{\dagger}_2}$]{#1}{#2}{#3}}

\newcommand{\dTdthr}[4][]{\dAmpdThr[$_{T^{\dagger}_3}$]{#1}{#2}{#3}{#4}}
\newcommand{\dTdthrs}[4][]{\dAmpdThr{#1}{#2}{#3}{#4}}
\newcommand{\dUdthr}[4][]{\dAmpdThr[$_{U^{\dagger}_3}$]{#1}{#2}{#3}{#4}}

%general excitations
\newcommand{\dT}[3][]{\dAmp[$_{T_#2}$]{#1}{#2}{#3}}
\newcommand{\dTs}[3][]{\dAmp{#1}{#2}{#3}}
\newcommand{\dU}[3][]{\dAmp[$_{U_#2}$]{#1}{#2}{#3}}
\newcommand{\dTv}[3][]{
\tikzset{exoper_line/.style={exvac_line_save}}
\dAmp{#1}{#2}{#3}
\tikzset{exoper_line/.style={exoper_line_save}} }

\newcommand{\dTd}[3][]{\dAmpD[$_{T^{\dagger}_#2}$]{#1}{#2}{#3}}
\newcommand{\dTds}[3][]{\dAmpD{#1}{#2}{#3}}
\newcommand{\dUd}[3][]{\dAmpD[$_{U^{\dagger}_#2}$]{#1}{#2}{#3}}
\newcommand{\dTdv}[3][]{
\tikzset{exoper_line/.style={exvac_line_save}}
\dAmpD{#1}{#2}{#3}
\tikzset{exoper_line/.style={exoper_line_save}} }

%Hamilton parts
\newcommand{\dF}[1]{\dHone[$_{F}$]{#1}}
\newcommand{\dFs}[1]{\dHone{#1}}
\newcommand{\dX}[1]{\dHone[$_{X}$]{#1}}

\newcommand{\dW}[2]{\dHtwo[$_{W}$]{#1}{#2}}
\newcommand{\dWs}[2]{\dHtwo{#1}{#2}}
\newcommand{\dXtwo}[2]{\dHtwo[$_{X}$]{#1}{#2}}

\newcommand{\dline}[3][]{
\ifthenelse{\isempty{#1}}
{\draw[ph_line](#2) -- (#3);}
{\draw[ph_line](#2) -- node[inner sep=0pt,minimum size=0pt,right = 0.5pt] {\footnotesize\slshape\sffamily #1} (#3);}
}
\newcommand{\dcurve}[3][]{
\ifthenelse{\isempty{#1}}
{\draw[ph_line](#2) to [bend right=30] (#3);}
{\draw[ph_line](#2) to [bend right=30] node[inner sep=0pt,minimum size=0pt,right = 0.5pt] {\footnotesize\slshape\sffamily #1} (#3);}
}
\newcommand{\dcurver}[3][]{
\ifthenelse{\isempty{#1}}
{\draw[ph_line](#2) to [bend left=30] (#3);}
{\draw[ph_line](#2) to [bend left=30] node[inner sep=0pt,minimum size=0pt,right = 0.5pt] {\footnotesize\slshape\sffamily #1} (#3);}
}
\newcommand{\dcurcur}[2]{\dcurve{#1}{#2} \dcurve{#2}{#1}}
\newcommand{\dmovex}[1]{\FPeval\xcoorH{clip(\xcoord+#1*0.25*\sc)}}
\newcommand{\dmoveT}[1]{\FPeval\xcoor{clip(\xcoord+#1*0.25*\sc)}}
\newcommand{\dmoveTd}[1]{\FPeval\xcoordg{clip(\xcoord+#1*0.25*\sc)}}
\newcommand{\dmovac}[1]{\FPeval\yvac{clip(\yvac+#1*0.25*\sc)}}
\newcommand{\dname}[1]{
\FPeval\xx{clip(\xcoord+1*\sc)} \FPeval\result{clip(\yvac+0.25*\sc)} 
\node at (\xx,\result) {#1};}
\newcommand{\dtext}[2]{\FPeval\xx{clip(\xcoord+#1*\sc)} \FPeval\result{clip((\ycoord+\yvac)/2)} 
\node at (\xx,\result) {#2};}

\newcommand{\dscale}[1]{\FPset\sc{#1} 
\FPeval\xcoord{clip(-1*\sc)} \FPeval\ycoord{clip(-1*\sc)} 
\FPset\xcoor{\xcoord} \FPeval\ycoor{clip(\ycoord+1*\sc)} 
\FPset\xcoordg{\xcoord}
\FPset\xcoorH{\xcoord}
\FPeval\yvac{clip(\yvac*\sc)} \FPeval\xvac{clip(\xvac*\sc)}}
% end of Diagram-package


% test:
% \begin{pspicture}(\xcoord,\ycoord)(3,3)
% \dTone{t1}
% \dTtwo{t21}{t22}
% \FPeval\xcoorbeg{clip(\xcoord+1)}
% \dV{vn1}{vn2}
% \dline{t1}{vn1}
% \dline{t1v1}{t1}
% \dline{t21}{vn2}
% \dline{t21}{t2v1}
% \dline{vn1}{vn1v2}
% \dline{vn2}{vn2v2}
% \dline{t22v1}{t22}
% \dline{t22}{t22v2}
% \end{pspicture}

%  
\end{lstlisting}

A diagram starts with {\bf \textbackslash bdiag} and ends with {\bf \textbackslash ediag}.

%\begin{table}[htbp]
 \begin{minipage}[b]{0.55\linewidth}\centering
  \begin{lstlisting}
%<\mu_1|\op F \op T_1 |0>
\bdiag
  \dmoveH{2}
  \dT{1}{t}
  \dF{f}
  \dline{tv1}{t}
  \dline{t}{f}
  \dline{f}{fv2}
\ediag 
  \end{lstlisting}
 \end{minipage}
 \begin{minipage}[b]{0.45\linewidth}\centering
    %<\mu_1|\op F \op T_1 |0>
    \bdiag
    \dmoveH{2}
    \dT{1}{t}
    \dF{f}
    \dline{tv1}{t}
    \dline{t}{f}
    \dline{f}{fv2}
    \ediag
 \end{minipage}
%\end{table}

For a symmetric diagram (with Hamilton-operator parts in the middle) use {\bf \textbackslash bdiags}.

%\begin{table}[htbp]
 \begin{minipage}[b]{0.55\linewidth}\centering
  \begin{lstlisting}
%<\mu_1|\op F \op T_1 |0>
\bdiags
  \dmoveH{2}
  \dT{1}{t}
  \dF{f}
  \dline{tv1}{t}
  \dline{t}{f}
  \dline{f}{fv2}
\ediag 
  \end{lstlisting}
 \end{minipage}
 \begin{minipage}[b]{0.45\linewidth}\centering
    %<\mu_1|\op F \op T_1 |0>
    \bdiags
    \dmoveH{2}
    \dT{1}{t}
    \dF{f}
    \dline{tv1}{t}
    \dline{t}{f}
    \dline{f}{fv2}
    \ediag
 \end{minipage}
%\end{table}

For a non-symmetric diagram with Hamilton-operator parts shifted down use {\bf \textbackslash bdiagd}.

%\begin{table}[htbp]
 \begin{minipage}[b]{0.55\linewidth}\centering
  \begin{lstlisting}
%<0|\op T_1^\dg \op F |\mu_1>
\bdiagd
  \dmoveH{2}
  \dTd{1}{td}
  \dF{f}
  \dline{tdv1}{td}
  \dline{td}{f}
  \dline{f}{fvd2}
\ediag
  \end{lstlisting}
 \end{minipage}
 \begin{minipage}[b]{0.45\linewidth}\centering
    %<0|\op T_1^\dg \op F |\mu_1>
    \bdiagd
    \dmoveH{2}
    \dTd{1}{td}
    \dF{f}
    \dline{tdv1}{td}
    \dline{td}{f}
    \dline{f}{fvd2}
    \ediag
 \end{minipage}
%\end{table}

You can scale the diagram by setting a number in the square brackets after {\bf \textbackslash bdiag}
or {\bf \textbackslash bdiags}.

%\begin{table}[htbp]
 \begin{minipage}[b]{0.55\linewidth}\centering
  \begin{lstlisting}
%<\mu_1|\op F \op T_1 |0>
\bdiags[1.5]
  \dmoveH{2}
  \dT{1}{t}
  \dF{f}
  \dline{tv1}{t}
  \dline{t}{f}
  \dline{f}{fv2}
\ediag 
  \end{lstlisting}
 \end{minipage}
 \begin{minipage}[b]{0.45\linewidth}\centering
    %<\mu_1|\op F \op T_1 |0>
    \bdiags[1.5]
    \dmoveH{2}
    \dT{1}{t}
    \dF{f}
    \dline{tv1}{t}
    \dline{t}{f}
    \dline{f}{fv2}
    \ediag
 \end{minipage}
%\end{table}

Or you can scale the diagram vertically and horizontally using

\myind{\bf \textbackslash dvscale}\{{\it scaling factor}\}

and

\myind{\bf \textbackslash dhscale}\{{\it scaling factor}\}
 
\section{Operators}

\subsection{Excitation and deexcitation operators}

\subsubsection{Coupled Cluster excitation/deexcitation operators}

The usual Coupled Cluster operator with label $T_{n}$ can be created using the following command:

\myind{\bf \textbackslash dT}$[${\it pert.order}$]$\{{\it exc.level}\}\{{\it node}\}

Node-names for vertices are generated as {\it node}1, {\it node}2, {\it node}3, ...

The best suitable nodes for external lines are also generated and called {\it node}1v1 and {\it node}1v2 
(for {\it node}1) etc

For the complex-conjugated counterpart (with label $T^{\dagger}_{n}$) use 

\myind{\bf \textbackslash dTd}$[${\it pert.order}$]$\{{\it exc.level}\}\{{\it node}\}

If the labels are not needed, use

\myind{\bf \textbackslash dTs}$[${\it pert.order}$]$\{{\it exc.level}\}\{{\it node}\}

\myind{\bf \textbackslash dTds}$[${\it pert.order}$]$\{{\it exc.level}\}\{{\it node}\}

%\begin{table}[htbp]
 \begin{minipage}[b]{0.55\linewidth}\centering
  \begin{lstlisting}
%<0|\op T_2^{(1)\dg} \op F \op T_1 |0>
\bdiags
\dmoveH{5}
\dmoveTd{1}
\dTds[1]{2}{td}
\dT{1}{t}
\dF{f}
\dcurcur{td1}{t1}
\dcurcur{td2}{f}
\ediag
  \end{lstlisting}
 \end{minipage}
 \begin{minipage}[b]{0.45\linewidth}\centering
%<0|\op T_2^{(1)\dg} \op F \op T_1 |0>
\bdiags
\dmoveH{5}
\dmoveTd{1}
\dTds[1]{2}{td}
\dT{1}{t}
\dF{f}
\dcurcur{td1}{t1}
\dcurcur{td2}{f}
\ediag
 \end{minipage}
%\end{table}

\subsubsection{Bare excitation/deexcitation operators}

One can draw the bare excitation/deexcitation operators explicitly and connect external lines to them.

For $\tau_{\mu_i}$ use

\myind{\bf \textbackslash dTv}$[${\it pert.order}$]$\{{\it exc.level}\}\{{\it node}\}

And for $\tau^{\dagger}_{\mu_i}$ use

\myind{\bf \textbackslash dTdv}$[${\it pert.order}$]$\{{\it exc.level}\}\{{\it node}\}

%\begin{table}[htbp]
 \begin{minipage}[b]{0.55\linewidth}\centering
  \begin{lstlisting}
%<\mu_2| \op F \op T_1 |0>
\bdiags
\dmoveH{5}
\dmoveTd{1}
\dTdv{2}{td}
\dT{1}{t}
\dF{f}
\dcurcur{td1}{t1}
\dcurcur{td2}{f}
\ediag
  \end{lstlisting}
 \end{minipage}
 \begin{minipage}[b]{0.45\linewidth}\centering
%<\mu_2| \op F \op T_1 |0>
\bdiags
\dmoveH{5}
\dmoveTd{1}
\dTdv{2}{td}
\dT{1}{t}
\dF{f}
\dcurcur{td1}{t1}
\dcurcur{td2}{f}
\ediag
 \end{minipage}
%\end{table}

\subsubsection{General excitation/deexcitation operators}

One can draw an excitation operator with a custom name using

\myind{\bf \textbackslash dAmp}$[${\it name}$]$\{{\it pert.order}\}\{{\it exc.level}\}\{{\it node}\}

And for an deexcitation operator use 

\myind{\bf \textbackslash dAmpD}$[${\it name}$]$\{{\it pert.order}\}\{{\it exc.level}\}\{{\it node}\}

%\begin{table}[htbp]
 \begin{minipage}[b]{0.55\linewidth}\centering
  \begin{lstlisting}
%<\Lambda_2| \op F \op T_1 |0>
\bdiags
\dmoveH{5}
\dmoveTd{1}
\dAmpD[$_{\Lambda_2}$]{}{2}{td}
\dT{1}{t}
\dF{f}
\dcurcur{td1}{t1}
\dcurcur{td2}{f}
\ediag
  \end{lstlisting}
 \end{minipage}
 \begin{minipage}[b]{0.45\linewidth}\centering
%<\Lambda_2| \op F \op T_1 |0>
\bdiags
\dmoveH{5}
\dmoveTd{1}
\dAmpD[$_{\Lambda_2}$]{}{2}{td}
\dT{1}{t}
\dF{f}
\dcurcur{td1}{t1}
\dcurcur{td2}{f}
\ediag
 \end{minipage}
%\end{table}

\subsubsection{Customize excitation/deexcitation operators}

It is possible to create custom (de)excitation operators. ${\mathbf U}_n $ and ${\mathbf U}^\dagger_n $ are 
available already (together with non-labeled versions dUs and dUds):

\myind{\bf \textbackslash dU}$[${\it pert.order}$]$\{{\it exc.level}\}\{{\it node}\}

\myind{\bf \textbackslash dUd}$[${\it pert.order}$]$\{{\it exc.level}\}\{{\it node}\}

There is also a transparent version of operators available:

\myind{\bf \textbackslash dTt}\{{\it exc.level}\}\{{\it node}\}

\myind{\bf \textbackslash dTtd}\{{\it exc.level}\}\{{\it node}\}

You can also create your own styles for operator lines (see Section \ref{sec:Styles}).

\subsection{Parts of Hamiltonian}

\subsubsection{Fock operator}

For Fock operator use

\myind{\bf \textbackslash dF}\{{\it node}\}

The best suitable nodes for external lines are called {\it node}v1 and {\it node}v2. 
Nodes for external lines going down are called {\it node}vd1 and {\it node}vd2.

For Fock operator without label use

\myind{\bf \textbackslash dFs}\{{\it node}\}

For a reverse Fock operator line (with $\times$ left) use {\bf \textbackslash dFr} or {\bf \textbackslash dFsr}.

\subsubsection{Fluctuation potential}

For fluctuation potential use

\myind{\bf \textbackslash dW}\{{\it node1}\}\{{\it node2}\}

The best suitable nodes for external lines are called {\it node1}v1 and {\it node1}v2, 
and {\it node2}v1 and {\it node2}v2.
Nodes for external lines going down are called {\it node1}vd1, {\it node1}vd2, 
{\it node2}vd1, and {\it node2}vd2.

For fluctuation potential without label use

\myind{\bf \textbackslash dWs}\{{\it node1}\}\{{\it node2}\}

A dressed version of fluctuation potential (with double lines) can be drawn
using {\bf \textbackslash dWbar} and {\bf \textbackslash dWbars} commands.

\subsubsection{Perturbations}

For one-electron perturbation use

\myind{\bf \textbackslash dX}\{{\it node}\}

({\bf \textbackslash dXr} for a reverse line)

For two-electron perturbation use

\myind{\bf \textbackslash dXtwo}\{{\it node1}\}\{{\it node2}\}

\subsubsection{Custom one- and two-electron parts}

For custom one-electron part use

\myind{\bf \textbackslash dHone}$[${\it name}$]$\{{\it node}\}

({\bf \textbackslash dHoner} for a reverse line)

and for two-electron part use

\myind{\bf \textbackslash dHtwo}$[${\it name}$]$\{{\it node1}\}\{{\it node2}\}

You can also create your own styles for operator lines (see Section \ref{sec:Styles}).

\subsubsection{Many-body operators}

It is possible to create more-than-two-body operators using 

\myind{\bf \textbackslash dHmany}$[${\it name}$]$\{{\it number of electrons}\}\{{\it node}\}

For three-body operators aliases {\bf \textbackslash dWthree}, {\bf \textbackslash dWthrees},
{\bf \textbackslash dWthreebar}, and {\bf \textbackslash dWthreebars} have been created, which
correspond to the {\bf \textbackslash dW} two-body operators. Note that for these operators
only main node name has to be given, like for excitation operators.

%\begin{table}[htbp]
 \begin{minipage}[b]{0.55\linewidth}\centering
  \begin{lstlisting}
\bdiag
\dWthree{w}
\dline{w1v1}{w1}
\dline{w1}{w1v2}
\dline{w2v1}{w2}
\dline{w2}{w2v2}
\dline{w3v1}{w3}
\dline{w3}{w3v2}
\ediag
  \end{lstlisting}
 \end{minipage}
 \begin{minipage}[b]{0.45\linewidth}\centering
\bdiag
\dWthree{w}
\dline{w1v1}{w1}
\dline{w1}{w1v2}
\dline{w2v1}{w2}
\dline{w2}{w2v2}
\dline{w3v1}{w3}
\dline{w3}{w3v2}
\ediag
 \end{minipage}

\subsubsection{Feynman vs Bartlett convention}

By default the interaction lines are drawn as dashed lines. If you prefer the 
usual Feynman's electromagnetic-interaction lines use

\myind{\bf \textbackslash dfeynman}

at the beginning of the diagram.

\subsection{Scaling of operator lines}

The size of operator lines can be influenced using

\myind{\bf \textbackslash dscaleop}\{{\it scaling factor}\}

\subsection{Effective Hamiltonian}

An effective Hamiltonian can be added using

\myind{\bf \textbackslash dHeff}[{\it name}]\{{\it pert.order}\}\{{\it number of electrons}\}\{{\it node}\}

Nodes above the operator are called {\it node}1a, {\it node}2a, etc., 
and below the operator -- {\it node}1b, {\it node}2b, etc.
The up-external-line nodes are called {\it node}1v1, {\it node}1v2, {\it node}2v1, {\it node}2v2, etc.,
and the down-external-line nodes -- {\it node}1vd1, {\it node}1vd2, {\it node}2vd1, {\it node}2vd2, etc.

The command

\myind{\bf \textbackslash dHeffs}[{\it pert.order}]\{{\it number of electrons}\}\{{\it node}\}

can be used to make the notation shorter.

The size of the Hamiltonian part can be changed using

\myind{\bf \textbackslash dheffsize}\{{\it size}\}

Default {\it size} is 6.
 
~\\
%\begin{table}[htbp]
 \begin{minipage}[b]{0.55\linewidth}\centering
  \begin{lstlisting}
\bdiags
\dHeff[$H_{\rm eff}$]{3}{2}{h}
\dline{h1a}{h1v1}
\dline{h1vd1}{h1b}
\dline{h2a}{h2v1}
\dline{h2v2}{h2a}
\ediags
  \end{lstlisting}
 \end{minipage}
 \begin{minipage}[b]{0.45\linewidth}\centering
\bdiags
\dHeff[$H_{\rm eff}$]{3}{2}{h}
\dline{h1a}{h1v1}
\dline{h1vd1}{h1b}
\dline{h2a}{h2v1}
\dline{h2v2}{h2a}
\ediags
 \end{minipage}
%\end{table}

%\begin{table}[htbp]
 \begin{minipage}[b]{0.55\linewidth}\centering
  \begin{lstlisting}
\bdiag
\dHeffs{3}{h}
\dmoveT{5}
\dU{1}{u}
\dline{h1a}{h1v1}
\dline{h1vd1}{h1b}
\dline{h2a}{h2v1}
\dline{h2v2}{h2a}
\dcurcur{u1}{h3b}
\ediag
  \end{lstlisting}
 \end{minipage}
 \begin{minipage}[b]{0.45\linewidth}\centering
\bdiag
\dHeffs{3}{h}
\dmoveT{5}
\dU{1}{u}
\dline{h1a}{h1v1}
\dline{h1vd1}{h1b}
\dline{h2a}{h2v1}
\dline{h2v2}{h2a}
\dcurcur{u1}{h3b}
\ediag
 \end{minipage}
%\end{table}

\section{Hole/Particle lines}

\subsection{Arrows}

By default arrows are placed at 62\% of the line. One can change it in the source code 
(``mark=at position'').
Alternatively to set arrows at the end of lines uncomment the corresponding ``ph-line-arrow-save''.

One can use 

\myind{\bf \textbackslash dnoarrow}

and

\myind{\bf \textbackslash darrow}

in order to switch off and on the arrows.

Double-headed arrows can be switched on using 

\myind{\bf \textbackslash ddoubleheadarrow}

%\begin{table}[htbp]
 \begin{minipage}[b]{0.55\linewidth}\centering
  \begin{lstlisting}
%<\mu_1|\op F \op T_1 |0>
\bdiags
  \dmoveH{2}
  \dT{1}{t}
  \dF{f}
  \dnoarrow
  \dline{tv1}{t}
  \ddoubleheadarrow
  \dline{t}{f}
  \darrow
  \dline{f}{fv2}
\ediag 
  \end{lstlisting}
 \end{minipage}
 \begin{minipage}[b]{0.45\linewidth}\centering
    %<\mu_1|\op F \op T_1 |0>
    \bdiags
    \dmoveH{2}
    \dT{1}{t}
    \dF{f}
    \dnoarrow
    \dline{tv1}{t}
    \ddoubleheadarrow
    \dline{t}{f}
    \darrow
    \dline{f}{fv2}
    \ediag
 \end{minipage}
%\end{table}


\subsection{Straight lines}

In order to connect two vertices with a straight h/p line use

\myind{\bf \textbackslash dline}$[${\it index}$]$\{{\it from-node}\}\{{\it to-node}\}

If {\it index} is given, it will be written to the right of the line.

\subsection{Curved lines}

In order to connect two vertices with a curved h/p line use

\myind{\bf \textbackslash dcurve}$[${\it index}$]$\{{\it from-node}\}\{{\it to-node}\}

If {\it index} is given, it will be written to the right of the line.

You can reverse the bend of the line using 

\myind{\bf \textbackslash dcurver}$[${\it index}$]$\{{\it from-node}\}\{{\it to-node}\}

instead.

You can draw a ``ring'' between two nodes using

\myind{\bf \textbackslash dcurcur}\{{\it from-node}\}\{{\it to-node}\}

\subsection{Intelligent lines through three vertices}

One can connect three vertices with an intelligent line. 

\myind{\bf \textbackslash dcurt}\{{\it from-node}\}\{{\it through-node}\}\{{\it to-node}\}

The most appropriate bend will be calculated automatically.

\subsection{Bubbles and oysters}
Internal connection of vertices can be achieved using  

\myind{\bf \textbackslash dbubble}\{{\it node}\}

and

\myind{\bf \textbackslash doyster}\{{\it from-node}\}\{{\it to-node}\}

or their counterparts with arrows going in the opposite direction {\bf \textbackslash dbubbler}
and {\bf \textbackslash doysterr}.

\section{Shifting operators}

Often in order to improve the diagram-look you have to shift the operator lines.
There are six shift-commands available:
\begin{itemize}
 \item shift excitation operators to the right:

\myind{\bf \textbackslash dmoveT}\{{\it shift}\}

 \item  shift deexcitation operators to the right:

\myind{\bf \textbackslash dmoveTd}\{{\it shift}\}

 \item  shift Hamiltonian-operators to the right:

\myind{\bf \textbackslash dmoveH}\{{\it shift}\}

  \item  shift excitation operators up (can also be used to shift up nodes of Hamiltonian lines):

\myind{\bf \textbackslash dvmoveT}\{{\it shift}\}

  \item  shift deexcitation operators up:

\myind{\bf \textbackslash dvmoveTd}\{{\it shift}\}

or 

\myind{\bf \textbackslash dmovac}\{{\it shift}\}

  \item  shift Hamiltonian-operators up:

\myind{\bf \textbackslash dvmoveH}\{{\it shift}\}

\end{itemize}

{\it shift}$=1$ in the horizontal direction corresponds to a shift of a half length of single excitation operator
(or of a quarter of doubles operator). 

Add origin to the diagram, if you want to shift operators in absolute coordinates:

\myind{\bf \textbackslash dorigin$[${\it x-coordinate\_of\_the\_origin}$]$}

Default x-coordinate of the origin is 0.

\section{Text in diagrams}

\myind{\bf \textbackslash dname}\{{\it text}\} -- write {\it text} (e.g. diagram name) over the diagram

\myind{\bf \textbackslash dtext}\{{\it shift}\}\{{\it text}\} -- write {\it text} in the diagram (with a horizontal shift).

\subsection{Named nodes}

Use operators

\myind{\bf \textbackslash dTone}$[${\it pert.order}$]$\{{\it node}\}\{{\it node name}\}

\myind{\bf \textbackslash dTtwo}$[${\it pert.order}$]$\{{\it node}\}\{{\it node1 name}\}\{{\it node2 name}\}

\myind{\bf \textbackslash dFn}\{{\it node}\}\{{\it node name}\}

\myind{\bf \textbackslash dWn}\{{\it node1}\}\{{\it node2}\}\{{\it node1 name}\}\{{\it node2 name}\}

to name individual nodes. Versions
{\bf \textbackslash dTsone},
{\bf \textbackslash dTstwo},
{\bf \textbackslash dTdone},
{\bf \textbackslash dTdtwo},
{\bf \textbackslash dTdsone},
{\bf \textbackslash dTdstwo},
{\bf \textbackslash dTdvone},
{\bf \textbackslash dTdvtwo},
{\bf \textbackslash dUone},
{\bf \textbackslash dUtwo},
{\bf \textbackslash dUsone},
{\bf \textbackslash dUstwo},
{\bf \textbackslash dFsn},
{\bf \textbackslash dFrn},
{\bf \textbackslash dFsrn},
{\bf \textbackslash dWsn}
are also available.

\section{Styles}\label{sec:Styles}

Change {\bf exoper-line-save}, {\bf exvac-line-save}, {\bf hoper-line-save}, {\bf heffoper-line-save}, 
{\bf ph-line-arrow-save}(or {\bf ph-line-noarrow-save}) in order to change excitation operator, 
bare excitation operator, H-operator, H$_{\rm eff}$-operator or h/p line styles, respectively.

You can create your own operator style for a custom operator (see how {\bf \textbackslash dU} or {\bf \textbackslash dTv} are 
defined).

\section{Generating external graphics}

Put 

\myind{\bf \textbackslash dsavediags}[{\it Name-for-diagrams}]\{{\it Real-name of the tex document}\}

in the preamble (e.g., after {\bf \textbackslash include\{ccdiag\}}).
The diagram-files will be called {\it Name-for-diagrams}1, {\it Name-for-diagrams}2, etc.

In order to save the diagram, put

\myind{\bf \textbackslash diagsav}[{\it Name for this diagram}] 

before it (with curly brackets around the diagram). The diagram will get the {\it Name for this diagram}
if it's given, or {\it Name-for-diagrams} with a number.

Now you can use provided script {\bf makediags} to generate all diagrams:

\myind{\bf makediags} {\it Real-name of the tex document.tex} [{\it number of diagram to start with}] [{\it number of diagram to stop with}]

If the {\it number of diagram to start with} is given, only diagrams from this number on will be recompiled.
If the {\it number of diagram to stop with} is given, only diagrams up to this number will be recompiled.

The resulting pdf files will be included by pdflatex 
instead of diagram-generation by later compilations (which can considerably speed up the compilation!) .

\subsection{Old way}

Alternatively, put 

\myind{\bf \textbackslash beginpgfgraphicnamed}\{{\it Name-of-diagram}\}

and

\myind{\bf \textbackslash endpgfgraphicnamed}

before and after a block of diagrams that can be put outside, and 

\myind{\bf \textbackslash pgfrealjobname}\{{\it Real-name of the tex document}\}

in the preamble (e.g., after {\bf \textbackslash include\{ccdiag\}}).


\myind{\bf \textbackslash pgfrealjobname}\{{\it Real-name of the tex document}\}

Use {\bf makediags} to compile the diagrams.

\section{Examples}
%\begin{table}[htbp]
 \begin{minipage}[b]{0.55\linewidth}\centering
  \begin{lstlisting}
\bdiag
\dmoveH{2}
\dT{1}{t}
\dmoveT{2}
\dT{2}{tt}
\dWs{wn1}{wn2}
\dline[\raisebox{0.5cm}{$i$}]{t1v1}{t1}
\dline{t1}{wn1}
\dline{wn1}{wn1v2}
\dline[$j$]{tt1v2}{tt1}
\dline{tt1}{wn2}
\dline{wn2}{wn2v1}
\dline[$k$]{tt2v1}{tt2}
\dline{tt2}{tt2v2}
\ediag
  \end{lstlisting}
 \end{minipage}
 \begin{minipage}[b]{0.45\linewidth}\centering
\bdiag
\dmoveH{2}
\dT{1}{t}
\dmoveT{2}
\dT{2}{tt}
\dWs{wn1}{wn2}
\dline[\raisebox{0.5cm}{$i$}]{t1v1}{t1}
\dline{t1}{wn1}
\dline{wn1}{wn1v2}
\dline[$j$]{tt1v2}{tt1}
\dline{tt1}{wn2}
\dline{wn2}{wn2v1}
\dline[$k$]{tt2v1}{tt2}
\dline{tt2}{tt2v2}
\ediag
 \end{minipage}
%\end{table}

%\begin{table}[htbp]
 \begin{minipage}[b]{0.55\linewidth}\centering
  \begin{lstlisting}
\bdiags
\dmoveT{1}
\dmoveTd{1}
\dU{1}{t}
\dTdv{1}{td}
\dW{w1}{w2}
\dcurt{t}{w1}{td}
\dcurt{td}{w2}{t}
\ediag
  \end{lstlisting}
 \end{minipage}
 \begin{minipage}[b]{0.45\linewidth}\centering
\bdiags
\dmoveT{1}
\dmoveTd{1}
\dU{1}{t}
\dTdv{1}{td}
\dW{w1}{w2}
\dcurt{t}{w1}{td}
\dcurt{td}{w2}{t}
\ediag
 \end{minipage}
%\end{table}

%\begin{table}[htbp]
 \begin{minipage}[b]{0.55\linewidth}\centering
  \begin{lstlisting}
\bdiags
\dmoveT{3}
\dmoveTd{3}
\dTd{2}{td}
\dUd{1}{ud}
\dFsr{f}
\dT{2}{t}
\dU{1}{u}
\dcurt{td1}{f}{t1}
\dcurve{t1}{td1}
\dcurve{td2}{t2}
\dline{t2}{ud}
\dcurver{ud}{u}
\dline{u}{td2}
\ediag
  \end{lstlisting}
 \end{minipage}
 \begin{minipage}[b]{0.45\linewidth}\centering
\bdiags
\dmoveT{3}
\dmoveTd{3}
\dTd{2}{td}
\dUd{1}{ud}
\dFsr{f}
\dT{2}{t}
\dU{1}{u}
\dcurt{td1}{f}{t1}
\dcurve{t1}{td1}
\dcurve{td2}{t2}
\dline{t2}{ud}
\dcurver{ud}{u}
\dline{u}{td2}
\ediag
 \end{minipage}
%\end{table}

%\begin{table}[htbp]
 \begin{minipage}[b]{0.55\linewidth}\centering
  \begin{lstlisting}
\bdiagd
\dmoveH{2}
\dTd{1}{t}
\dmoveTd{2}
\dTd{2}{tt}
\dWs{wn1}{wn2}
\dline[\raisebox{-0.5cm}{$i$}]{t1v1}{t1}
\dline{t1}{wn1}
\dline{wn1}{wn1vd2}
\dline[$j$]{tt1v2}{tt1}
\dline{tt1}{wn2}
\dline{wn2}{wn2vd1}
\dline[$k$]{tt2v1}{tt2}
\dline{tt2}{tt2v2}
\ediag
\end{lstlisting}
\end{minipage}
 \begin{minipage}[b]{0.45\linewidth}\centering
\bdiagd
\dmoveH{2}
\dTd{1}{t}
\dmoveTd{2}
\dTd{2}{tt}
\dWs{wn1}{wn2}
\dline[\raisebox{-0.5cm}{$i$}]{t1v1}{t1}
\dline{t1}{wn1}
\dline{wn1}{wn1vd2}
\dline[$j$]{tt1v2}{tt1}
\dline{tt1}{wn2}
\dline{wn2}{wn2vd1}
\dline[$k$]{tt2v1}{tt2}
\dline{tt2}{tt2v2}
\ediag
\end{minipage}
%\end{table}

\begin{minipage}[b]{0.55\linewidth}\centering
 \begin{lstlisting}
% transparent operators
\bdiagd
\dTtd{1}{t}
\dTt{1}{tt}
\dline{t1}{tt1}
\ediag
\end{lstlisting}
\end{minipage}
 \begin{minipage}[b]{0.45\linewidth}\centering
% transparent operators
\bdiagd
\dTtd{1}{t}
\dTt{1}{tt}
\dline{t1}{tt1}
\ediag
\end{minipage}

\begin{minipage}[b]{0.55\linewidth}\centering
 \begin{lstlisting}
% named nodes 
% and an external graphic "ExampleDiag"
\diagsav[ExampleDiag]{
\bdiags
\dmoveH{1}
\dmoveTd{6}
\dTdvtwo{t}{$\kappa$}{$\lambda$}
\dWsn{w1}{w2}{$\iota$}{$\kappa$}
\dTsone{tt}{$\iota$}
\dmoveT{2}
\dTstwo{ttt}{$\kappa$}{$\lambda$}
\dcurcur{tt}{w1}
\dcurt{ttt1}{w2}{t1}
\dcurver{t1}{ttt1}
\dcurcur{ttt2}{t2}
\ediag
}
\end{lstlisting}
\end{minipage}
 \begin{minipage}[b]{0.45\linewidth}\centering
% named nodes 
% and an external graphic "ExampleDiag"
\diagsav[ExampleDiag]{
\bdiags
\dmoveH{1}
\dmoveTd{6}
\dTdvtwo{t}{$\kappa$}{$\lambda$}
\dWsn{w1}{w2}{$\iota$}{$\kappa$}
\dTsone{tt}{$\iota$}
\dmoveT{2}
\dTstwo{ttt}{$\kappa$}{$\lambda$}
\dcurcur{tt}{w1}
\dcurt{ttt1}{w2}{t1}
\dcurver{t1}{ttt1}
\dcurcur{ttt2}{t2}
\ediag
}
\end{minipage}


\begin{minipage}[b]{0.55\linewidth}\centering
 \begin{lstlisting}
% bubble and oyster
% and the p/h counterparts 
\bdiags[2]
\dWs{w1}{w2}
\dbubble{w2}{1}
\ediag
\bdiags[2]
\dWs{w1}{w2}
\doyster{w1}{w2}
\ediag
\bdiags[2]
\dWs{w1}{w2}
\dbubbler{w2}{1}
\ediag
\bdiags[2]
\dWs{w1}{w2}
\doysterr{w1}{w2}
\ediag
\end{lstlisting}
\end{minipage}
 \begin{minipage}[b]{0.45\linewidth}\centering
% bubble and oyster
% and the p/h counterparts 
\bdiags[2]
\dWs{w1}{w2}
\dbubble{w2}{1}
\ediag
\bdiags[2]
\dWs{w1}{w2}
\doyster{w1}{w2}
\ediag
\bdiags[2]
\dWs{w1}{w2}
\dbubbler{w2}{1}
\ediag
\bdiags[2]
\dWs{w1}{w2}
\doysterr{w1}{w2}
\ediag
\end{minipage}

\subsection{DCD}
\begin{minipage}[b]{0.55\linewidth}\centering
\begin{lstlisting}
\bdiags[1.5]
\dvscale{0.5}
\dfeynman \dnoarrow
\dvmoveH{-2}
\dTdv{2}{td}
\dWbars{w1}{w2}
\dcurcur{w1}{td1}
\dcurcur{w2}{td2}
\ediag
\bdiags[1.5]
\dvscale{0.5}
\dfeynman \dnoarrow
\dTdv{2}{td}
\dmoveH{4.6}
\dFbars{f}
\dTs{2}{tt}
\dcurcur{tt1}{td1}
\dcurve{td2}{tt2}
\dcurt{tt2}{f}{td2}
\ediag
\bdiags[1.5]
\dvscale{0.5}
\dfeynman \dnoarrow
\dTdv{2}{td}
\dmoveH{1.2}
\dscaleop{0.7}
\dWs{w1}{w2}
\dscaleop{1}
\dTs{2}{tt}
\dcurve{td1}{tt1}
\dcurt{tt1}{w1}{td1}
\dcurve{tt2}{td2}
\dcurt{td2}{w2}{tt2}
\ediag
\bdiags[1.5]
\dvscale{0.5}
\dfeynman \dnoarrow
\dmoveH{4.5}
\dTdv{2}{td}
\dTs{2}{tt}
\dscaleop{0.37}
\dWs{w1}{w2}
\dscaleop{1}
\dcurcur{tt1}{td1}
\dcurt{tt2}{w1}{td2}
\dcurt{td2}{w2}{tt2}
\ediag
\end{lstlisting}
\end{minipage}
\begin{minipage}[b]{0.45\linewidth}\centering
\bdiags[1.5]
\dvscale{0.5}
\dfeynman \dnoarrow
\dvmoveH{-2}
\dTdv{2}{td}
\dWbars{w1}{w2}
\dcurcur{w1}{td1}
\dcurcur{w2}{td2}
\ediag
\bdiags[1.5]
\dvscale{0.5}
\dfeynman \dnoarrow
\dTdv{2}{td}
\dmoveH{4.6}
\dFbars{f}
\dTs{2}{tt}
\dcurcur{tt1}{td1}
\dcurve{td2}{tt2}
\dcurt{tt2}{f}{td2}
\ediag
\bdiags[1.5]
\dvscale{0.5}
\dfeynman \dnoarrow
\dTdv{2}{td}
\dmoveH{1.2}
\dscaleop{0.7}
\dWs{w1}{w2}
\dscaleop{1}
\dTs{2}{tt}
\dcurve{td1}{tt1}
\dcurt{tt1}{w1}{td1}
\dcurve{tt2}{td2}
\dcurt{td2}{w2}{tt2}
\ediag
\bdiags[1.5]
\dvscale{0.5}
\dfeynman \dnoarrow
\dmoveH{4.5}
\dTdv{2}{td}
\dTs{2}{tt}
\dscaleop{0.37}
\dWs{w1}{w2}
\dscaleop{1}
\dcurcur{tt1}{td1}
\dcurt{tt2}{w1}{td2}
\dcurt{td2}{w2}{tt2}
\ediag
\end{minipage}
\end{document}
